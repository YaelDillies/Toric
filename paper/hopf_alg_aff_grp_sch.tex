% !TeX program = xelatex
% !TEX options = --shell-escape paper/hopf_alg_aff_grp_sch.tex
\documentclass{article}

\title{Correspondence of Hopf algebras and affine group schemes in Lean 4}
\author{Yaël Dillies, Michał Mrugała, Andrew Yang\footnote{All authors contributed equally to this work.}}
\date{\today}

\usepackage{amsmath}
\usepackage{amsthm}
\usepackage{amssymb}
\usepackage{enumerate}
\usepackage{mathtools}
\usepackage{graphicx}
\usepackage{tikz}
\usepackage{mathrsfs}
\usepackage{parskip}
\usepackage{relsize}
\usepackage[hypcap=true]{caption}
\usepackage[shortlabels]{enumitem}
\usepackage{hyperref}

\usepackage[capitalise,noabbrev]{cleveref}
\usepackage{nameref}
\usepackage[margin=1in,a4paper]{geometry}

%%% Lean listings setup %%%

\usepackage{fontspec}
\usepackage{minted}

% JuliaMono has all the symbols needed for Lean code
\setmonofont[Scale=MatchLowercase]{JuliaMono}[
  Extension=.ttf,
  UprightFont=*-Regular,
  BoldFont=*-Bold,
  ItalicFont=*-RegularItalic,
  BoldItalicFont=*-BoldItalic,
  RawFeature={-calt},
]

% autogobble lets us indent the lean code to distinguish it from the latex
\setminted{frame=single}
\newmintinline[lean]{lean}{breaklines,breakbefore={. },breakafter={_}}
\newminted[leancode]{lean}{autogobble}

% Configure the appearance of minted.
% We adjust the spacing around minted environments, which stretches badly by default
\fvset{listparameters=\setlength{\topsep}{0pt}\setlength{\parsep}{0pt}}
\AtBeginEnvironment{minted}{\vspace{4pt}}
\renewcommand{\floatpagefraction}{.99}
\renewcommand{\bottomfraction}{.7}

%%% Lean listings setup %%%

% Theorems
\newtheorem{thm}{Theorem}[section]
\newtheorem{lem}[thm]{Lemma}
\newtheorem{prop}[thm]{Proposition}
\newtheorem{corollary}[thm]{Corollary}
\newtheorem{dfn}[thm]{Definition}
\newtheorem{eg}[thm]{Example}
\newtheorem{egs}[thm]{Examples}
\newtheorem{conj}{Conjecture}
\newtheorem{prob}[thm]{Problem}
\newtheorem{question}{Question}

% Command redirections
\let\P\oldP
\let\oldemptyset\emptyset
\let\emptyset\varnothing

% Letter shorthands
\newcommand{\C}{\mathbb C}
\newcommand{\bbE}{\mathbb E}
\newcommand{\F}{\mathbb F}
\newcommand{\K}{\mathbb K}
\newcommand{\N}{\mathbb N}
\newcommand{\Q}{\mathbb Q}
\newcommand{\R}{\mathbb R}
\newcommand{\Z}{\mathbb Z}
\newcommand{\mcA}{\mathcal A}
\newcommand{\mcB}{\mathcal B}
\newcommand{\mcC}{\mathcal C}
\newcommand{\mcD}{\mathcal D}
\newcommand{\mcE}{\mathcal E}
\newcommand{\mcF}{\mathcal F}
\newcommand{\mcG}{\mathcal G}
\newcommand{\mcH}{\mathcal H}
\newcommand{\mcM}{\mathcal M}
\newcommand{\mcN}{\mathcal N}
\newcommand{\mcO}{\mathcal O}
\newcommand{\mcP}{\mathcal P}
\newcommand{\mcQ}{\mathcal Q}
\newcommand{\mcR}{\mathcal R}
\newcommand{\mcS}{\mathcal S}
\newcommand{\mcT}{\mathcal T}
\newcommand{\mcU}{\mathcal U}
\newcommand{\mcV}{\mathcal V}
\newcommand{\eps}{\varepsilon}
\newcommand{\Eps}{\mathcal E}


\begin{document}


\maketitle


\begin{abstract}
  We formalise the correspondence between affine group schemes and Hopf algebras in Lean 4.
  We show that the category of affine group schemes is equivalent to the category of commutative Hopf algebras over a field.
  As an application, we construct the perfect pairing of characters and cocharacters.
  Our work is largely a reproduction of \cite{CrazyAffine}, with the major difference being that we work with group objects as defined by the usual diagrams as opposed to presheaves of groups.
  Our code is made available as part of Mathlib, the Lean 4 library of formalised mathematics.
\end{abstract}


\section{Introduction}\label{sec:intro}


Algebraic groups are a central object of study in algebraic geometry. TODO: Expand

Hopf algebras ... TODO: Expand

Since we have no use for not necessarily commutative rings, we drop henceforth the ``commutative'' prefix in front of ``ring'', ``algebra'', ``bialgebra'', ``Hopf algebra''.
We keep ``Comm'' in the names of our categories, for consistency with the Lean code.

Lean is an interactive theorem prover and functional programming language. TODO: Expand

Here is a sample of Lean code, claiming that composition of morphisms is associative:
\begin{leancode}
  variable {𝒞 : Type u} [Category.{v} 𝒞] {W X Y Z : 𝒞}

  lemma comp_assoc (f : W ⟶ X) (g : X ⟶ Y) (h : Y ⟶ Z) : f ≫ (g ≫ h) = (f ≫ g) ≫ h := ...
\end{leancode}

\lean{variable} is a keyword to introduce variables.
The combination \lean{{𝒞 : Type u} [Category.{v} 𝒞]} translates to ``Let $\mcC$ be a category with objects and morphisms in the $u$-th and $v$-th universes respectively''.
We will not have much use of universes and therefore will explain them no further.
All there is to know is that the categories relevant to the present paper have $u = v + 1$, corresponding to the fact that their objects form a proper class and their hom sets $X \to Y$ truly are sets.
\lean{{W X Y Z : 𝒞}} declares $W, X, Y, Z$ as objects in the category $\mcC$, and \lean{(f : W ⟶ X) (g : X ⟶ Y) (h : Y ⟶ Z)} declares morphisms between them.
\lean{f ≫ g} represents composition of morphisms, with the twist that the morphism $f$ is applied first, as opposed to the usual function composition notation \lean{g ∘ f}.
\lean{...} is not valid Lean syntax, but instead a placeholder we use throughout this paper to elide irrelevant details.
The full code for code snippets in each section is available on GitHub. See \url{https://github.com/YaelDillies/Toric/tree/master/HopfAlgAffGrpSchPaper}.

Our motivation for this work is not correctness.
Certainly, the material we formalise is part of the standard algebraic geometry curriculum and is correct without a doubt.
Our goal is instead to clear the prerequisites to more advanced topics, such as toric varieties.
In fact, we started off formalising toric varieties within the Toric project
\footnote{Available on GitHub. See \url{https://yaeldillies.github.io/Toric/}.}
before quickly realising that we first needed a solid theory of algebraic tori, which itself is most conveniently expressed in the language of algebraic groups.
Simultaneously, algebraic groups are needed for Kevin Buzzard's ongoing project to formalise Fermat's Last Theorem in Lean.
\footnote{Available on GitHub. See \url{https://imperialcollegelondon.github.io/FLT/}.}

The paper is structured as follows:
\cref{sec:informal} provides the mathematical background necessary to understanding the formalisation.
\cref{sec:hopf,sec:grp,sec:yoneda} illustrate design decisions we made, specific painpoints we identified and how we circumvented them, all grouped according to the relevant mathematical concepts.
These developments culminate in the definition of the perfect pairing between characters and cocharacters of an algebraic group, as described in \cref{sec:char-cochar}.


\section{Informal description of the correspondence}\label{sec:informal}


TODO


\section{Hopf algebras}\label{sec:hopf}


TODO(Michał)


\section{Group objects}\label{sec:grp}


TODO(Yaël)


\section{The Yoneda embedding}\label{sec:yoneda}


TODO(Andrew)


\section{The pairing of characters and cocharacters}\label{sec:char-cochar}


TODO


\section{Conclusion}\label{sec:conclusion}


TODO


\bibliographystyle{plain}
\bibliography{./hopf_alg_aff_grp_sch}

\end{document}
