% !TeX program = xelatex
% !TEX options = --shell-escape paper/hopf_alg_aff_grp_sch.tex
\documentclass{article}

\title{Correspondence of Hopf algebras and affine group schemes in Lean 4}
\author{Yaël Dillies, Michał Mrugała, Andrew Yang}
\date{\today}

\usepackage{amsmath}
\usepackage{amsthm}
\usepackage{amssymb}
\usepackage{enumerate}
\usepackage{fancyhdr}
\usepackage{mathtools}
\usepackage{graphicx}
\usepackage{tikz}
\usepackage{mathrsfs}
\usepackage{parskip}
\usepackage{relsize}
\usepackage[hypcap=true]{caption}
\usepackage[shortlabels]{enumitem}
\usepackage{hyperref}

\usepackage[capitalise,nameinlink,noabbrev]{cleveref}
\usepackage{nameref}
\usepackage[margin=1.5in,a4paper]{geometry}

%%% Lean listings setup %%%

\usepackage{fontspec}
\usepackage{minted}

% JuliaMono has all the symbols needed for Lean code
\setmonofont[Scale=MatchLowercase]{JuliaMono}[
  Extension=.ttf,
  UprightFont=*-Regular,
  BoldFont=*-Bold,
  ItalicFont=*-RegularItalic,
  BoldItalicFont=*-BoldItalic,
  RawFeature={-calt},
]

% autogobble lets us indent the lean code to distinguish it from the latex
\setminted{frame=single}
\newmintinline[lean]{lean}{breaklines,breakbefore={. },breakafter={_}}
\newminted[leancode]{lean}{autogobble}

% Configure the appearance of minted.
% We adjust the spacing around minted environments, which stretches badly by default
\fvset{listparameters=\setlength{\topsep}{0pt}\setlength{\parsep}{0pt}}
\AtBeginEnvironment{minted}{\vspace{4pt}}
\renewcommand{\floatpagefraction}{.99}
\renewcommand{\bottomfraction}{.7}

%%% Lean listings setup %%%

% Theorems
\newtheorem{thm}{Theorem}[section]
\newtheorem{lem}[thm]{Lemma}
\newtheorem{prop}[thm]{Proposition}
\newtheorem{corollary}[thm]{Corollary}
\newtheorem{dfn}[thm]{Definition}
\newtheorem{eg}[thm]{Example}
\newtheorem{egs}[thm]{Examples}
\newtheorem{conj}{Conjecture}
\newtheorem{prob}[thm]{Problem}
\newtheorem{question}{Question}

% Command redirections
\let\P\oldP
\let\oldemptyset\emptyset
\let\emptyset\varnothing

% Letter shorthands
\newcommand{\C}{\mathbb C}
\newcommand{\bbE}{\mathbb E}
\newcommand{\F}{\mathbb F}
\newcommand{\K}{\mathbb K}
\newcommand{\N}{\mathbb N}
\newcommand{\Q}{\mathbb Q}
\newcommand{\R}{\mathbb R}
\newcommand{\Z}{\mathbb Z}
\newcommand{\mcA}{\mathcal A}
\newcommand{\mcB}{\mathcal B}
\newcommand{\mcC}{\mathcal C}
\newcommand{\mcD}{\mathcal D}
\newcommand{\mcE}{\mathcal E}
\newcommand{\mcF}{\mathcal F}
\newcommand{\mcG}{\mathcal G}
\newcommand{\mcH}{\mathcal H}
\newcommand{\mcM}{\mathcal M}
\newcommand{\mcN}{\mathcal N}
\newcommand{\mcO}{\mathcal O}
\newcommand{\mcP}{\mathcal P}
\newcommand{\mcQ}{\mathcal Q}
\newcommand{\mcR}{\mathcal R}
\newcommand{\mcS}{\mathcal S}
\newcommand{\mcT}{\mathcal T}
\newcommand{\mcU}{\mathcal U}
\newcommand{\mcV}{\mathcal V}
\newcommand{\eps}{\varepsilon}
\newcommand{\Eps}{\mathcal E}


\begin{document}


\fancyfoot[L]{All authors contributed equally to this work.}
\maketitle
\thispagestyle{fancy}
\renewcommand{\footrulewidth}{0.4pt}


\begin{abstract}
  We formalise the correspondence between affine group schemes and Hopf algebras in Lean 4.
  We show that the category of affine group schemes is equivalent to the category of commutative Hopf algebras over a field.
  As an application, we construct the perfect pairing of characters and cocharacters.
  Our work is largely a reproduction of \cite{CrazyAffine}, with the major difference being that we work with group objects as defined by the usual diagrams as opposed to presheaves of groups.
  Our code is made available as part of Mathlib, the Lean 4 library of formalised mathematics.
\end{abstract}


\section{Introduction}\label{sec:intro}


TODO


\section{Hopf algebras}\label{sec:hopf}


TODO(Michał)


\section{Group objects}\label{sec:grp}


TODO(Yaël)


\section{The Yoneda embedding}\label{sec:yoneda}


TODO(Andrew)


\section{The pairing of characters and cocharacters}\label{sec:char-cochar}


TODO


\section{Conclusion}\label{sec:conclusion}


TODO


\bibliographystyle{plain}
\bibliography{./hopf_alg_aff_grp_sch}

\end{document}
