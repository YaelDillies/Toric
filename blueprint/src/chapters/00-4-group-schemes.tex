\section{Group Schemes}


\subsection{Correspondence between Hopf algebras and affine group schemes}


We want to show that Hopf algebras correspond to affine group schemes.
This can easily be done categorically assuming both categories on either side are defined thoughtfully.
However, the categorical version will not be workable with if we do not also have links to the non-categorical notions.
Therefore, one solution would be to build the left, top and right edges of the following
diagram so that the bottom edge can be obtained by composing the three:
%   Cogrp Mod_R   ⥤        Grp Sch_{Spec R}
%       ↑ ↓                        ↓
% R-Hopf algebras → Affine group schemes over Spec R


\subsubsection{Bundling/unbundling Hopf algebras}


We have already done the left edge in the previous section.


\subsubsection{Spec of a Hopf algebra}


Now let's do the top edge.


\begin{proposition}[Sliced adjoint functors]
  \label{0-slice-adj}
  \uses{}
  % \lean{}
  % \leanok

  If $a : F \vdash G$ is an adjunction between $F : C \to D$ and $G : D \to C$ and $X : C$, then there is an adjunction between $F / X : C / X \to D / F(X)$ and $G / X : D / F(X) \to C / X$.
\end{proposition}
\begin{proof}
  \uses{}
  % \leanok

  See https://ncatlab.org/nlab/show/sliced+adjoint+functors+--+section.
\end{proof}


\begin{proposition}[Limit-preserving functors lift to over categories]
  \label{0-over-lim}
  \uses{}
  \lean{CategoryTheory.Limits.PreservesLimitsOfShape.overPost, CategoryTheory.Limits.PreservesLimitsOfSize.overPost, CategoryTheory.Limits.PreservesFiniteProducts.overPost}
  \leanok

  Let $J$ be a shape (i.e. a category). Let $\widetilde J$ denote the category which is the same as $J$, but has an extra object $*$ which is terminal.
  If $F : C \to D$ is a functor preserving limits of shape $\widetilde J$, then the obvious functor $C / X \to D / F(X)$ preserves limits of shape $J$.
\end{proposition}
\begin{proof}
  \uses{}
  \leanok

  Extend a functor $K\colon  J \to C / X$ to a functor $\widetilde K\colon \widetilde J \to C$, by letting $\widetilde K (*) = X$.
\end{proof}


\begin{proposition}[Fully faithful product-preserving functors lift to group objects]
  \label{0-full-faithful-grp}
  \uses{}
  \lean{CategoryTheory.Functor.Faithful.mapGrp, CategoryTheory.Functor.Full.mapGrp}
  \leanok

  If a finite-products-preserving functor $F : C \to D$ is fully faithful, then so is $\Grp(F) : \Grp C \to \Grp D$.
\end{proposition}
\begin{proof}
  \uses{}
  \leanok

  Faithfulness is immediate.

  For fullness, assume $f : F(G) \to F(H)$ is a morphism. By fullness of $F$, find $g : G \to H$ such that $F(g) = f$. $g$ is a morphism because we can pull back each diagram from $D$ to $C$ along $F$ which is faithful.
\end{proof}


\begin{definition}[Spec as a functor on algebras]
  \label{0-spec-alg}
  \uses{}
  \lean{algSpec}
  \leanok

  Spec is a contravariant functor from the category of $R$-algebras to the category of schemes over $\Spec_R$.
\end{definition}


\begin{proposition}[Spec as a functor on algebras is fully faithful]
  \label{0-full-faithful-spec-alg}
  \uses{0-spec-alg}
  \lean{algSpec.instPreservesLimits, algSpec.instFull, algSpec.instFaithful, algSpec.fullyFaithful}
  \leanok

  Spec is a fully faithful contravariant functor from the category of $R$-algebras to the category of schemes over $\Spec_R$, preserving all limits.
\end{proposition}
\begin{proof}
  \uses{0-slice-adj, 0-over-lim}
  \leanok

  $\Spec : \Ring \to \Sch$ is a fully faithful contravariant functor which preserves all limits, hence so is $\Spec : \Ring_R \to \AffSch_{\Spec R}$ by Proposition \ref{0-over-lim} (alternatively, by Proposition \ref{0-slice-adj}).
\end{proof}


\begin{definition}[Spec as a functor on Hopf algebras]
  \label{0-spec-hopf}
  \uses{0-full-faithful-spec-alg}
  \lean{hopfSpec}
  \leanok

  Spec is a fully faithful contravariant functor from the category of $R$-algebras to the category of group schemes over $\Spec_R$.
\end{definition}


\begin{proposition}[Spec as a functor on Hopf algebras is fully faithful]
  \label{0-full-faithful-spec-hopf}
  \uses{0-spec-alg}
  \lean{hopfSpec.instFull, hopfSpec.instFaithful, hopfSpec.fullyFaithful}
  \leanok

  Spec is a fully faithful contravariant functor from the category of $R$-Hopf algebras to the category of group schemes over $\Spec_R$.
\end{proposition}
\begin{proof}
  \uses{0-full-faithful-grp, 0-full-faithful-spec-alg}
  \leanok

  $\Spec : \Ring_R \to \Sch_{\Spec R}$ is a fully faithful contravariant functor preserving all limits according to Proposition \ref{0-spec-alg}, therefore $\Spec : \Hopf_R \to \GrpSch_{\Spec R}$ too is fully faithful according to \ref{0-full-faithful-grp}.
\end{proof}


\subsection{Essential image of Spec on Hopf algebras}


Finally, let's do the right edge.


\begin{proposition}[Essential image of a sliced functor]
  \label{0-ess-image-over}
  \uses{}
  \lean{CategoryTheory.Functor.essImage_overPost}
  \leanok

  If $F : C \to D$ is a full functor between cartesian-monoidal categories, then $F / X : C / X \hom D / F(X)$ has the same essential image as $F$.
\end{proposition}
\begin{proof}
  \uses{}
  \leanok

  Transfer all diagrams.
\end{proof}


\begin{proposition}[Equivalences lift to group object categories]
  \label{0-grp-equiv}
  \uses{}
  \lean{CategoryTheory.Equivalence.mapGrp}
  \leanok

  If $e : C \backsimeq D$ is an equivalence of cartesian-monoidal categories, then $\Grp(e) : \Grp(C) \backsimeq \Grp(D)$ too is an equivalence of categories.
\end{proposition}
\begin{proof}
  \uses{}
  \leanok

  Transfer all diagrams.
\end{proof}


\begin{proposition}[Essential image of a functor on group objects]
  \label{0-ess-image-grp}
  \uses{}
  \lean{CategoryTheory.Functor.essImage_mapGrp}
  \leanok

  If $F : C \to D$ is a fully faithful functor between cartesian-monoidal categories, then $\Grp(F) : \Grp(C) \hom \Grp(D)$ has the same essential image as $F$.
\end{proposition}
\begin{proof}
  \uses{0-grp-equiv}
  % \leanok

  Transfer all diagrams.
\end{proof}


\begin{proposition}[Essential image of Spec on algebras]
  \label{0-ess-image-spec-alg}
  \uses{0-spec-alg}
  \lean{essImage_algSpec}
  \leanok

  The essential image of $\Spec : \Ring_R \to \Sch_{\Spec R}$ is precisely affine schemes over $\Spec R$.
\end{proposition}
\begin{proof}
  \uses{0-ess-image-over}
  \leanok

  Direct consequence of Proposition \ref{0-ess-image-over}.
\end{proof}


\begin{proposition}[Essential image of Spec on Hopf algebras]
  \label{0-ess-image-spec-hopf}
  \uses{0-spec-hopf}
  \lean{essImage_hopfSpec}
  \leanok

  The essential image of $\Spec : \Hopf_R \to \GrpSch_{\Spec R}$ is precisely affine group schemes over $\Spec R$.
\end{proposition}
\begin{proof}
  \uses{0-ess-image-grp, 0-ess-image-spec-alg, 0-full-faithful-spec-alg}
  \leanok

  Direct consequence of Propositions \ref{0-ess-image-grp} and \ref{0-ess-image-spec-alg}.
\end{proof}


\subsection{Diagonalisable groups}


\begin{definition}
  \label{0-spec-grp-alg}
  \uses{0-spec-hopf}
  \lean{hopfSpec, MonoidAlgebra}
  \leanok

  For a commutative group $G$ we define $D_R(G)$ as the spectrum $\Spec R[G]$ of the group algebra $R[G]$.
\end{definition}


\begin{definition}
  \label{0-diag}
  \uses{0-spec-grp-alg}
  \lean{AlgebraicGeometry.Scheme.IsDiagonalisable}
  % \leanok

  An algebraic group $G$ over $\Spec R$ is {\bf diagonalisable} if it is isomorphic to $D_R(G)$ for some commutative group $G$.
\end{definition}


\begin{theorem}
  \label{0-diag-iff-grp-like-span}
  \uses{0-spec-grp-alg, 0-diag}
  \lean{AlgebraicGeometry.Scheme.isDiagonalisable_iff_span_isGroupLikeElem_eq_top}
  \leanok

  An algebraic group $G$ over a field $k$ is diagonalizable if and only if group-like elements span $\Gamma(G)$.
\end{theorem}
\begin{proof}
  \uses{0-grp-like-lin-indep}
  % \leanok

  See Theorem 12.8 in \cite{Milne_2017}.
\end{proof}


\begin{theorem}
  \label{0-full-faithful-spec-grp-alg}
  \uses{0-spec-grp-alg}
  \lean{specCommGrpAlg.fullyFaithful, specCommGrpAlg.instFull, specCommGrpAlg.instFaithful}
  \leanok

  For a field $k$, $D_k$ is a fully faithful contravariant functor from the category of commutative groups to the category of group schemes over $\Spec k$.
\end{theorem}
\begin{proof}
  \uses{0-full-faithful-spec-hopf, 0-full-faithful-grp-alg}
  \leanok

  Compose Propositions \ref{0-full-faithful-spec-hopf} and \ref{0-full-faithful-grp-alg}.

  Also see Theorem 12.9(a) in \cite{Milne_2017}. See SGA III Exposé VIII for a proof that works for $R$ an arbitrary commutative ring in place of $k$.
\end{proof}
