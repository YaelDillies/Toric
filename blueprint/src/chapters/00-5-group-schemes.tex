\section{Group Schemes}


\subsection{Correspondence between Hopf algebras and affine group schemes}


We want to show that Hopf algebras correspond to affine group schemes.
This can easily be done categorically assuming both categories on either side are defined thoughtfully.
However, the categorical version will not be workable with if we do not also have links to the non-categorical notions.
Therefore, one solution would be to build the left, top and right edges of the following
diagram so that the bottom edge can be obtained by composing the three:
%   Cogrp Mod_R   ⥤        Grp Sch_{Spec R}
%       ↑ ↓                        ↓
% R-Hopf algebras → Affine group schemes over Spec R


\subsubsection{Bundling/unbundling Hopf algebras}


We have already done the left edge in the previous section.


\subsubsection{Spec of a Hopf algebra}


Now let's do the top edge.


\begin{definition}[Spec as a functor on algebras]
  \label{0-spec-alg}
  \uses{}
  \lean{algSpec}
  \leanok

  Spec is a contravariant functor from the category of $R$-algebras to the category of schemes over $\Spec_R$.
\end{definition}


\begin{proposition}[Spec as a functor on algebras is fully faithful]
  \label{0-full-faithful-spec-alg}
  \uses{0-spec-alg}
  \lean{algSpec.instPreservesLimits, algSpec.instFull, algSpec.instFaithful, algSpec.fullyFaithful}
  \leanok

  Spec is a fully faithful contravariant functor from the category of $R$-algebras to the category of schemes over $\Spec_R$, preserving all limits.
\end{proposition}
\begin{proof}
  \uses{0-slice-adj, 0-over-lim}
  \leanok

  $\Spec : \Ring \to \Sch$ is a fully faithful contravariant functor which preserves all limits, hence so is $\Spec : \Ring_R \to \AffSch_{\Spec R}$ by Proposition \ref{0-over-lim} (alternatively, by Proposition \ref{0-slice-adj}).
\end{proof}


\begin{definition}[Spec as a functor on Hopf algebras]
  \label{0-spec-hopf}
  \uses{0-full-faithful-spec-alg}
  \lean{hopfSpec}
  \leanok

  Spec is a contravariant functor from the category of $R$-Hopf algebras to the category of group schemes over $\Spec_R$.
\end{definition}


\begin{proposition}[Spec as a functor on Hopf algebras is fully faithful]
  \label{0-full-faithful-spec-hopf}
  \uses{0-spec-hopf}
  \lean{hopfSpec.instFull, hopfSpec.instFaithful, hopfSpec.fullyFaithful}
  \leanok

  Spec is a fully faithful contravariant functor from the category of $R$-Hopf algebras to the category of group schemes over $\Spec_R$.
\end{proposition}
\begin{proof}
  \uses{0-full-faithful-grp, 0-full-faithful-spec-alg}
  \leanok

  $\Spec : \Ring_R \to \Sch_{\Spec R}$ is a fully faithful contravariant functor preserving all limits according to Proposition \ref{0-spec-alg}, therefore $\Spec : \Hopf_R \to \GrpSch_{\Spec R}$ too is fully faithful according to \ref{0-full-faithful-grp}.
\end{proof}


\begin{proposition}[Spec sends cocommutative Hopf algebras to commutative group schemes]
  \label{0-spec-cocomm-hopf}
  \uses{0-spec-hopf}
  \lean{AlgebraicGeometry.Scheme.asOver.instIsComm_Mon, AlgebraicGeometry.Scheme.asOver.instCommGrp_Class}
  \leanok

  If $A$ is a cocommutative Hopf algebra over $R$, then $\Spec A$ is a commutative group scheme.
\end{proposition}
\begin{proof}
  \uses{}
  \leanok

  Diagrams are the same up to identifying $\Spec (A \otimes A)$ with $\Spec A \otimes \Spec A$.
\end{proof}


\subsection{Essential image of Spec on Hopf algebras}


Finally, let's do the right edge.


\begin{proposition}[Essential image of Spec on algebras]
  \label{0-ess-image-spec-alg}
  \uses{0-spec-alg}
  \lean{essImage_algSpec}
  \leanok

  The essential image of $\Spec : \Ring_R \to \Sch_{\Spec R}$ is precisely affine schemes over $\Spec R$.
\end{proposition}
\begin{proof}
  \uses{0-ess-image-over}
  \leanok

  Direct consequence of Proposition \ref{0-ess-image-over}.
\end{proof}


\begin{proposition}[Essential image of Spec on Hopf algebras]
  \label{0-ess-image-spec-hopf}
  \uses{0-spec-hopf}
  \lean{essImage_hopfSpec}
  \leanok

  The essential image of $\Spec : \Hopf_R \to \GrpSch_{\Spec R}$ is precisely affine group schemes over $\Spec R$.
\end{proposition}
\begin{proof}
  \uses{0-ess-image-grp, 0-ess-image-spec-alg, 0-full-faithful-spec-alg}
  \leanok

  Direct consequence of Propositions \ref{0-ess-image-grp} and \ref{0-ess-image-spec-alg}.
\end{proof}


\subsection{Diagonalisable groups}


\begin{definition}[The diagonalisable group scheme functor]
  \label{0-diag}
  \uses{0-grp-alg, 0-hopf-alg-equiv-cogrp-alg, 0-spec-hopf}
  \lean{hopfSpec, MonoidAlgebra}
  \leanok

  For a commutative group $G$ we define $D_R(G)$ as the spectrum $\Spec R[G]$ of the group algebra $R[G]$.
\end{definition}


\begin{definition}[Diagonalisable group schemes]
  \label{0-is-diag}
  \uses{0-diag}
  \lean{AlgebraicGeometry.Scheme.IsDiagonalisable}
  % \leanok

  An algebraic group $G$ over $\Spec R$ is {\bf diagonalisable} if it is isomorphic to $D_R(G)$ for some commutative group $G$.
\end{definition}


\begin{theorem}
  \label{0-diag-iff-grp-like-span}
  \uses{0-diag, 0-is-diag}
  \lean{AlgebraicGeometry.Scheme.isDiagonalisable_iff_span_isGroupLikeElem_eq_top}
  \leanok

  An algebraic group $G$ over a field $k$ is diagonalizable if and only if $\Gamma(G)$ is spanned by its group-like elements.
\end{theorem}
\begin{proof}
  \uses{0-grp-like-lin-indep}
  % \leanok

  See Theorem 12.8 in \cite{Milne_2017}.
\end{proof}


\begin{theorem}
  \label{0-full-faithful-diag}
  \uses{0-diag}
  \lean{specCommGrpAlg.fullyFaithful, specCommGrpAlg.instFull, specCommGrpAlg.instFaithful}
  \leanok

  Let $R$ be a domain.
  The functor $G \rightsquigarrow \Spec R[G]$ from the category of groups to the category of group schemes over $\Spec R$ is fully faithful.
\end{theorem}
\begin{proof}
  \uses{0-full-faithful-spec-hopf, 0-full-faithful-grp-alg}
  \leanok

  Compose Propositions \ref{0-full-faithful-spec-hopf} and \ref{0-full-faithful-grp-alg}.

  Also see Theorem 12.9(a) in \cite{Milne_2017}. See SGA III Exposé VIII for a proof that works for $R$ an arbitrary commutative ring.
\end{proof}


\subsection{The Torus}


\begin{definition}[The split torus]
  \label{0-torus}
  \lean{AlgebraicGeometry.Scheme.SplitTorus, AlgebraicGeometry.Scheme.SplitTorus.instCanonicallyOver}
  \leanok

  The split torus $\Gm^n$ over a scheme $S$ is the pullback of
  $\Spec \Z[x_1^{\pm 1}, \dotsc, x_n^{\pm 1}]$ along the unique map $S \to \Spec \Z$.
\end{definition}


\begin{lemma}[The split torus over $\Spec R$]
  \label{0-torus-spec}
  \uses{0-torus}
  \lean{AlgebraicGeometry.Scheme.splitTorusIsoSpec, AlgebraicGeometry.Scheme.splitTorusIsoSpecOver}
  \leanok

  The split torus over $\Spec R$ is isomorphic to $\Spec R[x_1^{\pm 1}, \dotsc, x_n^{\pm 1}]$.
\end{lemma}
\begin{proof}
  \uses{}
  % \leanok

  Ask any toddler on the street.
\end{proof}


\begin{lemma}[Diag is a group isomorphism on hom sets]
  \label{0-diag-hom}
  \uses{0-diag}
  % \lean{}
  % \leanok

  Let $R$ be a domain.
  The functor $G \rightsquigarrow \Spec R[G]$ from the category of groups to the category of group schemes over $\Spec R$ is a group isomorphism on hom sets.
\end{lemma}
\begin{proof}
  \uses{0-full-faithful-diag, 0-full-faithful-grp-hom}
  % \leanok

  Toddlers and streets by Lemmas \ref{0-full-faithful-diag} and \ref{0-full-faithful-grp-hom}.
\end{proof}


\begin{definition}[Characters of a group scheme]
  \label{0-char}
  \uses{0-torus}
  \lean{AlgebraicGeometry.Scheme.char}
  \leanok

  For a group scheme $G$ over $S$, the {\bf character lattice} of $G$ is
  \[
    X(G) := \Hom_{\mathsf{GrpSch}_S}(G, \Gm).
  \]
  An element of $X(G)$ is called a {\bf character}.
\end{definition}


\begin{definition}[Cocharacters of a group scheme]
  \label{0-cochar}
  \uses{0-torus}
  \lean{AlgebraicGeometry.Scheme.cochar}
  \leanok

  For a group scheme $G$ over $S$, the {\bf cocharacter lattice} of $G$ is
  \[
    X^*(G) := \Hom_{\mathsf{GrpSch}_S}(\Gm, G).
  \]
  An element of $X^*(G)$ is called a {\bf cocharacter} or {\bf one-parameter subgroup}.
\end{definition}


\begin{proposition}[Character lattice of a diagonalisable group scheme]
  \label{0-char-diag}
  \uses{0-char}
  \lean{AlgebraicGeometry.Scheme.charGrpAlg}
  \leanok

  Let $R$ be a domain and $G$ be a commutative group.
  Then $X(\Spec R[G]) = G$.
\end{proposition}
\begin{proof}
  \uses{0-full-faithful-diag}
  % \leanok

  By Propositions \ref{0-torus-spec} and \ref{0-full-faithful-diag} in turn, we have
  \[
    X(G) = \Hom_{\mathsf{GrpSch}}(G, \Gm) = \Hom(k[\Z], k[G]) = \Hom(\Z, G) = G.
  \]
\end{proof}


\begin{proposition}[Cocharacter lattice of a diagonalisable group scheme]
  \label{0-cochar-diag}
  \uses{0-cochar}
  \lean{AlgebraicGeometry.Scheme.cocharGrpAlg}
  \leanok

  Let $R$ be a domain and $G$ be a commutative group.
  Then $X^*(\Spec R[G]) = \Hom(G, \Z)$.
\end{proposition}
\begin{proof}
  \uses{0-full-faithful-diag}
  % \leanok

  By Propositions \ref{0-torus-spec} and \ref{0-full-faithful-diag} in turn, we have
  \[
    X^*(G) = \Hom_{\mathsf{GrpSch}}(\Gm, G) = \Hom(k[G], k[\Z]) = \Hom(G, \Z).
  \]
\end{proof}


\begin{proposition}[Character lattice of the torus]
  \label{0-char-torus}
  \uses{0-char}
  % \lean{}
  % \leanok

  Let $G$ be a torus of dimension $n$ over a domain $R$.
  Then $X(G) = \Z^n$.
\end{proposition}
\begin{proof}
  \uses{0-full-faithful-diag, 0-torus-spec}
  % \leanok

  Immediate from Propositions \ref{0-torus-spec} and \ref{0-char-diag}.
\end{proof}


\begin{proposition}[Cocharacter lattice of the torus]
  \label{0-cochar-torus}
  \uses{0-cochar}
  % \lean{}
  % \leanok

  Let $G$ be a torus of dimension $n$ over a domain $R$.
  Then $X^*(G) = \Hom(\Z^n, \Z)$.
\end{proposition}
\begin{proof}
  \uses{0-full-faithful-diag, 0-torus-spec}
  % \leanok

  Immediate from Propositions \ref{0-torus-spec} and \ref{0-cochar-diag}.
\end{proof}


\begin{definition}[The character-cocharacter pairing]
  \label{0-char-cochar-pairing}
  \uses{0-char, 0-cochar}
  \lean{AlgebraicGeometry.Scheme.charPairing, AlgebraicGeometry.Scheme.charPairingInt}
  % \leanok

  Let $R$ be a domain and $G$ a group scheme over $\Spec R$.
  Then there is a $\Z$-valued perfect pairing between $X(G)$ and $X^*(G)$.
\end{definition}
