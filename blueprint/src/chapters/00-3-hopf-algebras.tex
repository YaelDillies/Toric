\section{Hopf algebras}


\subsection{Group-like elements}


\begin{definition}[Group-like elements]
  \label{0-grp-like}
  \uses{}
  \lean{Bialgebra.IsGroupLikeElem}
  \leanok

  An element $a$ of a bi-algebra $A$ is \emph{group-like} if it is a unit and $\Delta(a) = a \ox a$, where $\Delta$ is the comultiplication map.
\end{definition}


\begin{lemma}[The image of a group-like element by the counit is $1$]
  \label{0-grp-like-counit}
  \uses{0-grp-like}
  \lean{Bialgebra.IsGroupLikeElem.counit_eq_one}
  \leanok

  If $A$ is a bialgebra, then the counit of $A$ sends every group-like element of
  $A$ to $1$.
\end{lemma}
\begin{proof}
  \uses{}
  \leanok

  Let $\Delta$ be the comultiplication map and $\epsilon$ be the counit map.
  Let $a$ be a group-like element of $A$.
  Using the coalgebra axioms and the fact that $\Delta(x) = x\otimes x$, we get:
  \[x \otimes 1 = (\mathbf{1}_A\otimes\epsilon)(\Delta(x))= x \otimes\epsilon(x).\]
  As $x$ is invertible, this implies that $1\otimes 1 = 1\otimes\epsilon(x)$ in
  $R\otimes_R A$. But the algebra map from $R$ to $A$ is injective because $A$
  is a bialgebra (hence the counit is a left inverse of this map), so $1=\epsilon(x)$.
\end{proof}


\begin{proposition}[Group-like elements form a group]
  \label{0-grp-like-grp}
  \uses{0-grp-like}
  \lean{Bialgebra.GroupLike.instGroup, Bialgebra.GroupLike.instCommGroup}
  \leanok

  Group-like elements of a bi-algebra $A$ form a group under multiplication.
\end{proposition}
\begin{proof}
  \uses{}
  \leanok

  Check that group-like elements are closed under unit, multiplication and inverses.
\end{proof}


\begin{lemma}[Bialgebra homs preserve group-like elements]
  \label{0-grp-like-map}
  \uses{0-grp-like}
  \lean{Bialgebra.IsGroupLikeElem.map}
  \leanok

  Let $f : A \to B$ be a bi-algebra hom. If $a \in A$ is group-like, then $f(a)$ is group-like too.
\end{lemma}
\begin{proof}
  \uses{}
  \leanok

  $a$ is a unit, so $f(a)$ is a unit too. Then
  \[
    f(a) \ox f(a) = (f \ox f)(\Delta_A(a)) = \Delta_B(f(a))
  \]
  so $f(a)$ is group-like.
\end{proof}


\begin{lemma}
  \label{0-grp-like-grp-alg-of}
  \uses{0-grp-like}
  \lean{MonoidAlgebra.isGroupLikeElem_of}
  \leanok

  If $R$ is a commutative semiring, $A$ is a Hopf algebra over $R$ and
  $G$ is a group, then every element of the image of $G$ in $A[G]$ is group-like.
\end{lemma}
\begin{proof}
  \uses{}
  \leanok

  This is an easy check.
\end{proof}


\begin{lemma}
  \label{0-grp-like-grp-alg-span}
  \uses{0-grp-like}
  \lean{MonoidAlgebra.span_isGroupLikeElem}
  \leanok

  If $R$ is a commutative semiring, $A$ is a Hopf algebra over $R$ and
  $G$ is a group, then the group-like elements in $A[G]$ span $A[G]$ as
  an $A$-module.

\end{lemma}
\begin{proof}
  \uses{0-grp-like-grp-alg-of}
  \leanok

  This follows immediately from \ref{0-grp-like-grp-alg-of}.
\end{proof}


\begin{lemma}[Independence of group-like elements]
  \label{0-grp-like-lin-indep}
  \uses{0-grp-like}
  \lean{Coalgebra.linearIndepOn_isGroupLikeElem}
  \leanok

  The group-like elements in a bialgebra $A$ over a field are linearly independent.
\end{lemma}
\begin{proof}
  \uses{0-tensor-lin-indep}
  \leanok

  See Lemma 4.23 in \cite{Milne_2017}.
\end{proof}


\begin{lemma}[Group-like elements in a group algebra]
  \label{0-grp-like-grp-alg}
  \uses{0-grp-like}
  \lean{MonoidAlgebra.isGroupLikeElem_iff_mem_range_of}
  \leanok

  Let $k$ be a field.
  The group-like elements of $k[M]$ are exactly the image of $M$.
\end{lemma}
\begin{proof}
  \uses{0-grp-like-lin-indep}
  \leanok

  See Lemma 12.4 in \cite{Milne_2017}.
\end{proof}


\subsection{Diagonalizable bialgebras}


\begin{definition}[Diagonalizable bialgebras]
  \label{0-diag-bialg}
  \uses{}
  \lean{Bialgebra.IsDiagonalisable}
  \leanok

  A bialgebra is called diagonalizable if it is isomorphic to a group algebra.
\end{definition}

\begin{lemma}
  \label{0-diag-bialg-group-like-span}
  \uses{0-diag-bialg, 0-grp-like}
  \lean{Bialgebra.span_isGroupLikeElem_eq_top_of_isDiagonalisable}
  \leanok

  A diagonalizable bialgebra is spanned by its group-like elements.
\end{lemma}
\begin{proof}
  \uses{0-grp-like-grp-alg-span}
  \leanok

  This is true for a group algebra by \ref{0-grp-like-grp-alg-span}, and the
  property of being spanned by its group-like elements is preserves by
  isomorphisms of bialgebras.
\end{proof}


\begin{proposition}
  \label{0-bialg-bij-of-span-group-like}
  \uses{0-diag-bialg, 0-grp-like-grp}
  \lean{Bialgebra.liftGroupLikeBialgHom_bijective_of_span_isGroupLikeElem_eq_top}
  \leanok

  Let $A$ be a bialgebra over a field $k$, and let $G$ be the set of group-like
  elements of $A$ (which is a group by \ref{0-grp-like-grp}). If $A$ is generated
  by $G$, then the unique bialgebra morphism from $k[G]$ to $A$ sending each
  element of $G$ to iself is bijective.
\end{proposition}
\begin{proof}
  \uses{0-grp-like-lin-indep}
  \leanok

  This morphism is injective by the linear independence of group-like elements
  (\ref{0-grp-like-lin-indep}), and surjective because the group-like elements
  of $A$ span $A$ by assumption.
\end{proof}


\begin{corollary}
  \label{0-bialg-diag-of-span-group-like}
  \uses{0-diag-bialg, 0-grp-like}
  \lean{Bialgebra.isDiagonalisable.iff_span_isGroupLikeElem_eq_top}
  \leanok

  A bialgebra over a field is diagonalizable if and only if it is spanned by its
  group-like elements.
\end{corollary}
\begin{proof}
  \uses{0-diag-bialg-group-like-span, 0-bialg-bij-of-span-group-like}
  \leanok

  We know that a diagonalizable bialgebra is spanned by its group-like elements
  by \ref{0-diag-bialg-group-like-span}, and that a bialgebra over a field
  that is spanned by its group-like elements is diagonalizable by
  \ref{0-bialg-bij-of-span-group-like} (and by the fact that a bijective morphism
  of bialgebras is an isomorphism).
\end{proof}

Proposition \ref{0-bialg-bij-of-span-group-like} and Corollary
\ref{0-diag-iff-grp-like-span} are false over a general
commutative ring. Indeed, let $R$ be a commutative ring and let $G$ be a group.
Then the group-like elements of $R[G]$ correspond to locally constant maps
  from $Spec(R)$ to $G$ (with the discrete topology), hence they are of the
form $e_1 g_1+\cdots+e_r g_r$, with the $g_i$ in $G$ and $e_1,\ldots,e_r$
a family of pairwise orthogonal idempotent elements of $R$ that sum to $1$.
So $R[G]$ is not isomorphic to the group algebra over its group-like elements
unless $Spec(R)$ is connected. As for the corollary, a bialgebra of the form
$R_1[G_1]\times\cdots\times R_n[G_n]$, seen as a bialgebra over
$R_1\times\cdots\times R_n$, is generated by its group-like elements but not
diagonalizable.

It is likely that both results are still true if the base ring has a connected
spectrum.


\subsection{The group algebra functor}


\begin{proposition}[The antipode is a antihomomorphism]
  \label{0-hopf-antipode-antihom}
  \uses{}
  \lean{HopfAlgebra.antipode_mul_antidistrib, HopfAlgebra.antipode_mul_distrib}
  \leanok

  If $A$ is a $R$-Hopf algebra, then the antipode map $s : A \to A$ is anti-commutative, ie $s(a * b) = s(b) * s(a)$. If further $A$ is commutative, then $s(a * b) = s(a) * s(b)$.
\end{proposition}
\begin{proof}
  \uses{}
  % \leanok

  Any standard reference will have a proof.
\end{proof}


\begin{proposition}[Hopf algebras are cogroup objects in the category of algebras]
  \label{0-hopf-cogrp-alg}
  \uses{}
  \lean{grp_Class_op_commAlgOf, isMon_Hom_commAlgOfHom, bialgebra_unop, hopfAlgebra_unop, IsMon_Hom.toBialgHom}
  \leanok

  From a $R$-Hopf algebra, one can build a cogroup object in the category of $R$-algebras.

  From a cogroup object in the category of $R$-algebras, one can build a $R$-Hopf algebra.
\end{proposition}
\begin{proof}
  \uses{0-hopf-antipode-antihom}
  \leanok

  Turn the arrows around.
\end{proof}


\begin{definition}[The group algebra functor]
  \label{0-grp-alg}
  \uses{0-hopf-cogrp-alg}
  \lean{commGrpAlg}
  \leanok

  For a commutative ring $R$, we have a functor $G \rightsquigarrow R[G] : \Grp \to \Hopf_R$.
\end{definition}


\begin{proposition}[The group algebra functor is fully faithful]
  \label{0-full-faithful-grp-alg}
  \uses{0-grp-alg}
  \lean{MonoidAlgebra.mapDomainBialgHomMulEquiv, AddMonoidAlgebra.mapDomainBialgHomMulEquiv, commGrpAlg.fullyFaithful, commGrpAlg.instFull, commGrpAlg.instFaithful}
  \leanok

  For a field $K$, the functor $G \rightsquigarrow K[G]$ from the category of groups to the category of Hopf algebras over $K$ is fully faithful.
\end{proposition}
\begin{proof}
  \uses{0-grp-like-grp-alg, 0-grp-like-map}
  \leanok

  It is clearly faithful.
  Now for the full part, if $f : K[G] \to K[H]$ is a Hopf algebra hom, then we get a series of maps
  \[
    G \simeq \text{ group-like elements of } R[G] \to \text{ group-like elements of } R[H] \simeq H
  \]
  and each map separately is clearly multiplicative.
\end{proof}
