\section{Introduction to Affine Toric Varieties}


\subsection{The Torus}


\begin{definition}[The split torus]
  \label{1-1-torus}
  \lean{AlgebraicGeometry.Scheme.SplitTorus, AlgebraicGeometry.Scheme.SplitTorus.instCanonicallyOver}
  \leanok

  The split torus $\Gm^n$ over a scheme $S$ is the pullback of
  $\Spec \Z[x_1^{\pm 1}, \dotsc, x_n^{\pm 1}]$ along the unique map $S \to \Spec \Z$.
\end{definition}


\begin{lemma}[The split torus over $\Spec R$]
  \label{1-1-torus-spec}
  \uses{1-1-torus}
  \lean{AlgebraicGeometry.Scheme.splitTorusIsoSpec, AlgebraicGeometry.Scheme.splitTorusIsoSpecOver}
  \leanok

  The split torus over $\Spec R$ is isomorphic to $\Spec(R[x_1^{\pm 1}, \dotsc, x_n^{\pm 1}])$.
\end{lemma}
\begin{proof}
  \uses{}
  % \leanok

  Ask any toddler on the street.
\end{proof}


\begin{definition}[Characters of a group scheme]
  \label{1-1-char}
  \uses{1-1-torus}
  \lean{AlgebraicGeometry.Scheme.Over.char}
  \leanok

  For a group scheme $G$ over $S$, the {\bf character lattice} of $G$ is
  \[
    X(G) := \Hom_{\mathsf{GrpSch}_S}(G, \Gm).
  \]
  An element $X(G)$ is (unsurprisingly) called a {\bf character}.
\end{definition}

\begin{proposition}[Character lattice of the torus]
  \label{1-1-char-torus}
  \uses{1-1-char}
  % \lean{}
  % \leanok

  Characters of the torus over a field $k$ are isomorphic to $\Z^n$. $X(\Gm^n) = \Z^n$.
\end{proposition}
\begin{proof}
  \uses{0-full-faithful-spec-grp-alg, 1-1-torus-spec}
  % \leanok

  By Propositions \ref{1-1-torus-spec} and \ref{0-full-faithful-spec-grp-alg} in turn, we have
  \[
    X(\Gm^n) = \Hom_{\mathsf{GrpSch}}(\Gm^n, \Gm) = \Hom(k[\Z], k[\Z^n]) = \Hom(\Z, \Z^n) = \Z^n.
  \]
\end{proof}


\begin{proposition}[The image of a torus is a torus]
  \label{1-1-1-group-hom-subtorus} % Proposition 1.1.1(a)
  \uses{1-1-torus}
  % \lean{}
  % \leanok

  Let $T_1$ and $T_2$ be split tori over a field $k$ and let $\Phi: T_1 \to T_2$ be a homomorphism,
  then $\Phi$ factors as
  \[
    T_1 \xrightarrow{\Phi} T_2 = T_1 \xrightarrow{\phi} T \xrightarrow{\iota} T_2,
  \]
  where $T$ is a split torus, $\iota$ is a closed subgroup embedding and $\phi$ is an fpqc homomorphism.
\end{proposition}
\begin{proof}
  \uses{0-full-faithful-spec-grp-alg, 1-1-torus-spec}
  % \leanok
  Let $M_1=X(T_1), M_2=X(T_2)$. Define $M$ to be the image of the homomorphism $M_2 \to M_1$
  corresponding to $\Phi$ and take $T = D_k(M)$. The homomorphisms $\iota,\phi$ correspond to the
  canonical quotient map $M_2 \to M$ and the canonical inclusion $M \to M_1$ respectively.
  Hence $\Phi = \iota\circ\phi$.

  $M$ is a subgroup of a finitely-generated free abelian group $M_1$,
  hence itself a finitely-generated free abelian group. Thus $T$ is a split torus.

  $\iota$ is a closed embedding since the corresponding ring map is a quotient map with kernel
  generated by the kernel of $M_2 \to M_1$.

  $\phi$ is affine, hence quasi-compact. A collection of coset representatives for $M /M_1$
  gives a basis for $k[M]$ as a $k[M_1]$ module, hence $\phi$ is faithfully flat.
\end{proof}


\begin{proposition}[A subgroup of a torus is a torus]
  \label{1-1-1-subgroup-subtorus}
  \uses{1-1-torus}
  % \lean{}
  % \leanok

  Let $T$ be a split torus.
  If $H \subseteq T$ is an irreducible subgroup, then $H$ is a split torus.
\end{proposition}
\begin{proof}
  \uses{0-full-faithful-spec-grp-alg, 1-1-torus-spec}
  % \leanok

  Let $M = X(T), N=X(H)$. Since $H$ is a closed subscheme $M \to N$ is surjective,
  so $N$ is a finitely-generated abelian group. Since $H$ is irreducible it is connected,
  so $N$ is torsion-free, hence free. Thus $H$ is a split torus.
\end{proof}


\begin{definition}[The character eigenspace]
  \label{1-1-char-eigenspace}
  \uses{1-1-char}
  % \lean{}
  % \leanok

  For a finite dimensional representation of a torus $T$ on $W$, the {\bf character eigenspace} of a character $\chi \in X(T)$ is
  \[
    W_m = \{w\in W : t\cdot w = \chi(t)\text{ for all } t\in T \}.
  \]
\end{definition}


\begin{proposition}[Decomposition into character eigenspaces]
  \label{1-1-2-char-eigenspace-direct-sum}
  \uses{1-1-char-eigenspace}

  The space decomposes into the direct sum of the character eigenspaces.
\end{proposition}
\begin{proof}
  \uses{}
  % \leanok

  TODO
\end{proof}

\begin{definition}
  \label{1-1-cochar}
  \uses{1-1-torus}
  \lean{AlgebraicGeometry.Scheme.Over.cochar}
  \leanok

  For a group scheme $G$, the {\bf cocharacter lattice} of $G$ is $\Hom_{\mathsf{GrpSch}_S}(\Gm, G)$.
  An element is called a {\bf cocharacter} or {\bf one-parameter subgroup}.
\end{definition}

\begin{definition}[The character-cocharacter pairing]
  \label{1-1-char-cochar-pairing}
  \uses{1-1-char, 1-1-cochar}
  \lean{AlgebraicGeometry.Scheme.Over.charPairing}
  % \leanok

  Character lattice and one-parameter subgroup pairing.
\end{definition}


\begin{proposition}[Cocharacter lattice of the torus]
  \label{1-1-cochar-torus}
  \uses{1-1-cochar}
  % \lean{}
  % \leanok

  $N = \Hom(M, \Z) \cong \Z^n$. For $u \in N$ we write $\lambda^u$ for the corresponding cocharacter.
\end{proposition}
\begin{proof}
  \uses{1-1-char-torus, 1-1-char-cochar-pairing}
  % \leanok

  By Propositions \ref{1-1-torus-spec} and \ref{0-full-faithful-spec-grp-alg} in turn, we have
  \[
    \mathrm{cochar}(\Gm^n) = \Hom_{\mathsf{GrpSch}}(\Gm, \Gm^n) = \Hom(k[\Z^n], k[\Z]) = \Hom(\Z^n, \Z) \cong \Z^n.
  \]
\end{proof}


\subsection{The Definition of Affine Toric Variety}


\begin{definition}
  \label{1-1-3-aff-tor-var}
  \uses{1-1-torus}
  \lean{AlgebraicGeometry.ToricVariety}
  \leanok

  A {\bf toric variety} is a variety $X$ with
  \begin{itemize}
    \item an open embedding $T := (\bbC^\times)^n \hookrightarrow X$ with dense image
    \item such that the natural action $T \times T \to T$ of the torus on itself extends to an (algebraic) action $T \times X \to X$.
  \end{itemize}
\end{definition}


\subsection{Lattice Points}


\begin{definition}
  \label{1-1-phiA}
  \uses{1-1-char}
  % \lean{}
  % \leanok

  Given a finite set $\MCA = \{a_1, \dotsc, a_s\} \subseteq M$, define $\Phi_\MCA : T \to \mathbb{A}^s$ given by $\Phi_{\mathcal A} (t) = (\chi^{a_1} (t), \dotsc, \chi^{a_s} (t))$.
\end{definition}


\begin{definition}
  \label{1-1-7-ya}
  \uses{1-1-phiA}
  % \lean{}
  % \leanok

  $Y_\MCA$ is the (Zariski) closure of $\im \Phi_\MCA$ in $\mathbb A^s$.
\end{definition}


\begin{proposition}
  \label{1-1-8-aff-tor-var-ya}
  \uses{1-1-7-ya}
  \uses{1-1-char}

  Proposition 1.1.8
\end{proposition}
\begin{proof}
  \uses{1-1-1-group-hom-subtorus}
  % \leanok

  TODO
\end{proof}


\subsection{Toric Ideals}


\begin{proposition}
  \label{1-1-9-ideal-ya}
  \uses{1-1-7-ya}
  % \lean{}
  % \leanok

  The ideal of the affine toric variety $Y_\MCA$ is
  \[
    I(Y_\MCA) = \langle x^{\ell_+} - x^{\ell_-} | \ell \in L\rangle
  \]
\end{proposition}
\begin{proof}
  \uses{}
  % \leanok

  See \cite{Cox_2011}.
\end{proof}


\begin{definition}
  \label{1-1-10-lattice-ideal}
  \uses{}
  \lean{AddMonoidAlgebra.monoidIdeal}
  \leanok

  The ideal $I_L = \langle x^\alpha - x^\beta | \alpha, \beta \in \N^s \text{ and } \alpha - \beta \in L\rangle$ is called the {\bf lattice ideal} of the lattice $L \subseteq \Z^s$.

  A toric ideal is a prime lattice ideal.
\end{definition}


\begin{definition}
  \label{1-1-10-toric-ideal}
  \uses{1-1-10-lattice-ideal}
  \lean{AddMonoidAlgebra.IsToricIdeal}
  \leanok
  A {\bf toric ideal} is a prime lattice ideal.
\end{definition}


\begin{proposition}
  \label{1-1-11-toric-ideal-gen-binomial}
  \uses{1-1-10-toric-ideal}
  \lean{AddMonoidAlgebra.isToricIdeal_iff_exists_span_single_sub_single}
  % \leanok

  Proposition 1.1.11: an ideal is toric if and only if it's prime and generated by binomials $x^\alpha - x^\beta$.
\end{proposition}
\begin{proof}
  \uses{1-1-1-subgroup-subtorus, 1-1-9-ideal-ya}
  % \leanok

\end{proof}


\begin{proposition}[The spectrum of an affine monoid algebra is an affine toric variety]
  \label{1-1-14-aff-tor-var-spec-aff-mon-alg}
  \uses{0-aff-mon, 1-1-3-aff-tor-var}
  \lean{AffineToricVarietyFromMonoid.instToricVariety}
  \leanok

  If $S$ is an affine monoid, then $\Spec(\Bbbk[S])$ is an affine toric variety.
\end{proposition}
\begin{proof}
  \uses{0-aff-mon-alg-domain, 0-full-faithful-grp-alg, 1-1-torus-spec}
  % \leanok

  Identify the torus with $\Bbbk[x_1^{\pm1}, \dotsc, x_n^{\pm1}]$ using Lemma \ref{1-1-torus-spec}.
  $i$ induces a morphism $T \to \Spec(\Bbbk[S])$. It's an open embedding as $i$ gives the localization of $\Bbbk[S]$ at $\chi^{a_i}$, so $\im i$ is an affine open. It's dominant as $\Spec(\Bbbk[S])$ is integral and so is irreducible, and $\im i$ is open and nonempty, so dense. The torus action is given by the natural restriction of comultiplication on $\Bbbk[x_1^{\pm1}, \dotsc, x_n^{\pm1}]$ using Proposition \ref{0-full-faithful-grp-alg}.
\end{proof}


\begin{proposition}[The character lattice of the spectrum of an affine monoid algebra]
  \label{1-1-14-char-spec-aff-mon-alg}
  \uses{1-1-14-aff-tor-var-spec-aff-mon-alg, 1-1-char}
  % \lean{}
  % \leanok

  If $S$ is an affine monoid, then the character lattice of $\Spec(\Bbbk[S])$ is $\Z S$.
\end{proposition}
\begin{proof}
  \uses{}
  % \leanok

  It is what it is.
\end{proof}


\begin{proposition}
  \label{1-1-14-spec-aff-mon-alg-eq-ya}
  \uses{0-aff-mon, 1-1-3-aff-tor-var, 1-1-7-ya}
  % \lean{}
  % \leanok

  If $S$ is an affine monoid and $\mathcal A$ is a finite set generating $S$ as a monoid, then $\Spec(\Bbbk[S]) = Y_{\mathcal A}$.
\end{proposition}
\begin{proof}
  \uses{1-1-8-aff-tor-var-ya, 1-1-14-char-spec-aff-mon-alg}
  % \leanok

  We get a $\Bbbk$-algebra homomorphism $\pi : \Bbbk[x_1, \dotsc, x_s] \to \Bbbk[\Z S]$ given by $\mathcal A$; this induces a morphism $\Phi_{\mathcal A} : T \to \Bbbk^s$. The kernel of $\pi$ is the toric ideal of $Y_{\mathcal A}$ and $\pi$ is clearly surjective, so $Y_{\mathcal A} = \mathbb V(\ker(\pi)) = \Spec(\Bbbk[x_1, \dotsc, x_s] / \ker(\pi)) = \Spec(\bbC[S])$.
\end{proof}


\begin{definition}
  \label{torActOnAlg}
  \uses{1-1-torus}
  % \lean{}
  % \leanok

  Torus action on semigroup algebra
\end{definition}


\subsection{Equivalence of Constructions}


\begin{lemma}
  \label{lmm:1-1-16}
  \uses{torActOnAlg}
\end{lemma}
\begin{proof}
  \uses{1-1-2-char-eigenspace-direct-sum}
  % \leanok

\end{proof}


\begin{theorem}
  \label{thm:1-1-17}
  \uses{0-aff-mon, 1-1-3-aff-tor-var, 1-1-7-ya, 1-1-10-lattice-ideal}
  TFAE:
  \begin{enumerate}
    \item $V$ is an affine toric variety.
    \item $V = Y_{\mathcal A}$ for some finite $\mathcal A$.
    \item $V$ is an affine variety defined by a toric ideal.
    \item $V = \Spec \Bbbk[S]$ for an affine monoid $S$.
  \end{enumerate}
\end{theorem}
\begin{proof}
  \uses{torActOnAlg, 1-1-8-aff-tor-var-ya, 1-1-9-ideal-ya, 1-1-14-aff-tor-var-spec-aff-mon-alg, lmm:1-1-16}
  % \leanok

\end{proof}
