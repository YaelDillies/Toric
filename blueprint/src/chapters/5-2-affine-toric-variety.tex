\section{Affine toric varieties and affine monoids}\label{sec:aff-tor-var}


In this section, we construct affine toric varieties from affine monoids,
and show all affine toric varieties arise from affine monoids in this way.


\subsection{Toric varieties from affine monoids}


\begin{proposition}[The diagonalisable group scheme of an affine monoid algebra is an affine toric variety]
  \label{1-1-aff-tor-var-diag-aff-mon}
  \uses{0-diag, 5-1-tor-var}
  % \lean{}
  % \leanok

  Let $k$ be a field.
  Let $G$ be a finitely generated free abelian group.
  Let $M$ be an affine monoid with Grothendieck group $G$.
  Then $D_k(M)$ is an affine toric variety over $k$ with torus $D_k(G)$.
\end{proposition}
\begin{proof}
  \uses{0-aff-mon-alg-domain, 0-loc-mon-alg, 0-comap-mod, 0-diag-spec}
  % \leanok

  The map $D_k(G) \hookrightarrow D_k(M)$ is given by the embedding $M \hookrightarrow G$.

  We identify $D_k(G) \cong \Spec k[G], D_k(M) \cong \Spec k[M]$ using
  Proposition \ref{0-diag-spec}.

  By Proposition \ref{0-loc-mon-alg}, $k[G]$ is a localization of $k[M]$.
  Therefore the map $\Spec k[G] \hookrightarrow \Spec k[M]$ is an open embedding.

  This open embedding is dominant since $\Spec k[M]$ is irreducible as $k[M]$ is a domain
  (Proposition \ref{0-aff-mon-alg-domain}).

  The group action $D_k(G) \times D_k(M) \to D_k(M)$ comes
  from pulling back along $D_k(G) \hookrightarrow D_k(M)$
  the left action $D_k(M) \times D_k(M) \to D_k(M)$ using Proposition \ref{0-comap-mod}.
\end{proof}


\subsection{Essential surjectivity from affine monoids to affine toric varieties}


\begin{definition}[The character eigenspace]
  \label{1-1-char-eigenspace}
  \uses{0-char}
  % \lean{}
  % \leanok

  For a finite dimensional representation of a torus $T$ on $W$, the {\bf character eigenspace} of a character $\chi \in X(T)$ is
  \[
    W_m = \{w\in W : t\cdot w = \chi(t)\text{ for all } t\in T \}.
  \]
\end{definition}


\begin{proposition}[Decomposition into character eigenspaces]
  \label{1-1-2-char-eigenspace-direct-sum}
  \uses{1-1-char-eigenspace}

  The space decomposes into the direct sum of the character eigenspaces.
\end{proposition}
\begin{proof}
  \uses{}
  % \leanok

  TODO
\end{proof}


\begin{definition}
  \label{1-1-tor-act-alg}
  \uses{0-torus}
  % \lean{}
  % \leanok

  There is a torus action on the semigroup algebra $\bbC[M]$: given $t\in T_N$ and $f\in \bbC[M]$ define
  \[
    t \cdot f = (p \mapsto f(t^{-1}p)).
  \]
\end{definition}


\begin{lemma}
  \label{1-1-16-total-red}
  \uses{1-1-tor-act-alg}
    Let $A \subseteq \bbC[M]$ be a stable subspace, then
    \[
      A = \bigoplus_{\chi^m \in A} \bbC \cdot \chi^m.
    \]
\end{lemma}
\begin{proof}
  \uses{1-1-2-char-eigenspace-direct-sum}
  % \leanok

  TODO
\end{proof}


\begin{definition}[Characters of a toric variety]
  \label{5-2-char-tor-var}
  \uses{}
  % \lean{}
  % \leanok

  Let $k$ be a field.
  Let $T$ be a torus over $k$.
  Let $V$ be a toric variety with torus $k$.
  The \emph{characters} $X(V)$ of $V$ are defined as the intersection of $X(T)$
  with the image of the map $k[V] \to k[T]$ of coordinate rings
  induced by the embedding $T \hookrightarrow V$.
\end{definition}


\begin{proposition}[Characters of a toric variety are an affine monoid]
  \label{5-2-char-tor-var-aff-mon}
  \uses{5-2-char-tor-var}
  % \lean{}
  % \leanok

  Let $k$ be a field.
  Let $T$ be a torus over $k$.
  Let $V$ be a toric variety with torus $k$.
  Then $X(V)$ is an affine monoid.
\end{proposition}
\begin{proof}
  \uses{}
  % \leanok

  TODO
\end{proof}


\begin{theorem}[Affine toric varieties come from affine monoids]
  \label{5-1-aff-tor-var-iso-diag-aff-mon}
  \uses{0-aff-mon, 5-1-tor-var, 5-1-tor-hom, 1-1-7-ya, 1-1-10-lat-ideal}
  % \lean{}
  % \leanok

  Let $k$ be a field.
  Let $T$ be a torus over $k$.
  Let $V$ be a toric variety with torus $k$.
  Then there exists a torus isomorphism $V \cong D_k(X(V))$.
\end{theorem}
\begin{proof}
  \uses{1-1-tor-act-alg, 1-1-8-aff-tor-var-ya, 1-1-9-ideal-ya, 1-1-14-aff-tor-var-spec-aff-mon-alg, 1-1-16-total-red, 5-2-char-tor-var-aff-mon}
  % \leanok

  TODO
\end{proof}
