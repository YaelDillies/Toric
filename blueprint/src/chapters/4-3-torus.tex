\section{The torus}


\begin{definition}[The split torus]
  \label{0-torus}
  \lean{AlgebraicGeometry.Scheme.SplitTorus, AlgebraicGeometry.Scheme.SplitTorus.instCanonicallyOver}
  \leanok

  The split torus $\G_m^n$ over a scheme $S$ is the pullback of
  $\Spec \Z[x_1^{\pm 1}, \dotsc, x_n^{\pm 1}]$ along the unique map $S \to \Spec \Z$.
\end{definition}


\begin{lemma}[Diag is a group isomorphism on hom sets]
  \label{0-diag-hom}
  \uses{0-diag}
  \lean{AlgebraicGeometry.Scheme.diagHomEquiv}
  \leanok

  Let $R$ be a domain.
  The functor $G \rightsquigarrow \Spec R[G]$ from the category of groups to the category of group schemes over $\Spec R$ is a group isomorphism on hom sets.
\end{lemma}
\begin{proof}
  \uses{0-full-faithful-diag, 0-full-faithful-grp-hom}
  \leanok

  Toddlers and streets by Lemmas \ref{0-full-faithful-diag} and \ref{0-full-faithful-grp-hom}.
\end{proof}


\begin{definition}[Characters of a group scheme]
  \label{0-char}
  \uses{0-torus}
  \lean{AlgebraicGeometry.Scheme.char}
  \leanok

  For a group scheme $G$ over $S$, the {\bf character lattice} of $G$ is
  \[
    X(G) := \Hom_{\mathsf{GrpSch}_S}(G, \G_m).
  \]
  An element of $X(G)$ is called a {\bf character}.
\end{definition}


\begin{definition}[Cocharacters of a group scheme]
  \label{0-cochar}
  \uses{0-torus}
  \lean{AlgebraicGeometry.Scheme.cochar}
  \leanok

  For a group scheme $G$ over $S$, the {\bf cocharacter lattice} of $G$ is
  \[
    X^*(G) := \Hom_{\mathsf{GrpSch}_S}(\G_m, G).
  \]
  An element of $X^*(G)$ is called a {\bf cocharacter} or {\bf one-parameter subgroup}.
\end{definition}


\begin{proposition}[Character lattice of a diagonalisable group scheme]
  \label{0-char-diag}
  \uses{0-char}
  \lean{AlgebraicGeometry.Scheme.charDiag}
  \leanok

  Let $R$ be a domain and $G$ be a commutative group.
  Then $X(\Spec R[G]) = G$.
\end{proposition}
\begin{proof}
  \uses{0-full-faithful-diag}
  \leanok

  By Propositions \ref{0-diag-spec} and \ref{0-full-faithful-diag} in turn, we have
  \[
    X(G) = \Hom_{\mathsf{GrpSch}}(G, \G_m) = \Hom(k[\Z], k[G]) = \Hom(\Z, G) = G.
  \]
\end{proof}


\begin{proposition}[Cocharacter lattice of a diagonalisable group scheme]
  \label{0-cochar-diag}
  \uses{0-cochar}
  \lean{AlgebraicGeometry.Scheme.cocharDiag}
  \leanok

  Let $R$ be a domain and $G$ be a commutative group.
  Then $X^*(\Spec R[G]) = \Hom(G, \Z)$.
\end{proposition}
\begin{proof}
  \uses{0-full-faithful-diag}
  \leanok

  By Propositions \ref{0-diag-spec} and \ref{0-full-faithful-diag} in turn, we have
  \[
    X^*(G) = \Hom_{\mathsf{GrpSch}}(\G_m, G) = \Hom(k[G], k[\Z]) = \Hom(G, \Z).
  \]
\end{proof}


\begin{proposition}[Character lattice of the torus]
  \label{0-char-torus}
  \uses{0-char}
  \lean{AlgebraicGeometry.Scheme.charTorus}
  \leanok

  Let $G$ be a torus of dimension $n$ over a domain $R$.
  Then $X(G) = \Z^n$.
\end{proposition}
\begin{proof}
  \uses{0-full-faithful-diag, 0-diag-spec}
  \leanok

  Immediate from Propositions \ref{0-diag-spec} and \ref{0-char-diag}.
\end{proof}


\begin{proposition}[Cocharacter lattice of the torus]
  \label{0-cochar-torus}
  \uses{0-cochar}
  \lean{AlgebraicGeometry.Scheme.cocharTorus}
  \leanok

  Let $G$ be a torus of dimension $n$ over a domain $R$.
  Then $X^*(G) = \Hom(\Z^n, \Z)$.
\end{proposition}
\begin{proof}
  \uses{0-full-faithful-diag, 0-diag-spec}
  \leanok

  Immediate from Propositions \ref{0-diag-spec} and \ref{0-cochar-diag}.
\end{proof}


\begin{definition}[The character-cocharacter pairing]
  \label{0-char-cochar-pairing}
  \uses{0-char, 0-cochar}
  \lean{AlgebraicGeometry.Scheme.charPairing}
  \leanok

  Let $R$ be a domain and $G$ a group scheme over $\Spec R$.
  Then there is a $\Z$-valued perfect pairing between $X(G)$ and $X^*(G)$.
\end{definition}


\begin{proposition}[The character-cocharacter pairing is perfect]
  \label{0-char-cochar-pairing-perfect}
  \uses{0-char-cochar-pairing}
  \lean{AlgebraicGeometry.Scheme.isPerfPair_charPairing}
  \leanok

  The character-cocharacter pairing on a torus is perfect.
\end{proposition}
\begin{proof}
  \uses{}
  \leanok

  Transfer everything across the isos $X(\G_m^n) = \Z^n, X*(\G_m^n) = \Hom(\Z^n, \Z)$.
\end{proof}


\begin{proposition}[The image of a torus is a torus]
  \label{1-1-1-group-hom-torus} % Proposition 1.1.1(a)
  \uses{0-torus}
  % \lean{}
  % \leanok

  Let $R$ be a domain. Let $T$ be a split torus over $R$.
  Let $G$ be a diagonalisable group scheme over $R$ and let $\phi: T \to G$ be a homomorphism.
  Then the (scheme theoretic) image of $\phi$ is a split torus over $R$ and the maps
  \[
    T \xrightarrow{\hat{\phi}} \operatorname{im}\phi \xrightarrow{\iota} G
  \]
  are group homomorphisms, and $\hat{\phi}$ is fpqc.
  Furthermore, if $T = D_R(H), G = D_R(I), \phi = D_R(f)$
  for $H$ a finitely generated free abelian group, $I$ an abelian group, $f : I \to H$ a group hom,
  then $\operatorname{im} \phi \cong D_R(\operatorname{im}(f))$.
\end{proposition}
\begin{proof}
  \uses{0-full-faithful-diag, 0-diag-spec, 0-diag-closed-emb, 0-diag-aff-hom}
  % \leanok

  By fullness of $D_R$ (Proposition \ref{0-full-faithful-diag}),
  it's enough to handle the case where $T = D_R(H), G = D_R(I), \phi = D_R(f)$
  for $H$ a finitely generated free abelian group, $I$ an abelian group, $f : I \to H$ a group hom.

  Then $\operatorname{im}(f)$ is a subgroup of the finitely-generated free abelian group $H$,
  hence itself a finitely-generated free abelian group
  (since free is equivalent to torsion-free for finitely-generated abelian groups,
  and a subgroup of a torsion-free group is torsion-free).
\end{proof}


\begin{proposition}[A subgroup of a torus is a torus]
  \label{1-1-1-subgroup-torus}
  \uses{0-torus}
  % \lean{}
  % \leanok

  Let $R$ be a commutative ring of characteristic zero.
  Let $T$ be a split torus.
  If $H \subseteq T$ is a connected closed subgroup, then $H$ is a split torus.
\end{proposition}
\begin{proof}
  \uses{0-subgroup-diag, 0-full-faithful-diag, 0-diag-spec, 0-diag-tors}
  % \leanok

  By assumption, write $T \cong D_k[G]$ for $G$ a free abelian group.
  By Proposition \ref{0-subgroup-diag}, $H$ is a diagonalisable group scheme,
  say $H \cong D_k(I)$ for $I$ an abelian group.
  Since $H$ is a closed subscheme, the map $G \to I$ is surjective,
  so $I$ is a finitely-generated abelian group.
  Since $H$ is connected, Proposition \ref{0-diag-tors} says $I$ is torsion-free, hence free.
  Thus $H$ is a split torus.
\end{proof}

