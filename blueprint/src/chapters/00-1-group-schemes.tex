\subsection{Group schemes}

\begin{definition}
  \label{grp_like}
  A \emph{group-like} element for a Hopf algebra $A$ is
  an $a\in A^*$ such that $\Delta(a) = a \ox a$, where $\Delta$ is the comultiplication map.
\end{definition}

\begin{lemma}
  \label{grp_like_li}
  \uses{grp_like}
  The group-like elements in $A$ are linearly independent.
\end{lemma}
\begin{proof}
  See Lemma 4.23 in \cite{Milne_2017}.
\end{proof}

\begin{definition}
  \label{grp_alg_sch}
  For a commutative group $M$ we define $D(M)$ as the spectrum of the group algebra $\Spec R[M]$.
\end{definition}

\begin{proposition}
  \label{DM_func}
  \uses{grp_alg_sch}
  For every finitely generated commutative group $M$, the algebraic group $D(M)$
  represents the functor $R rightsquigarrow \Hom_{\mathsf{Grp}}(M,R^*)$.
\end{proposition}
\begin{proof}
  See Proposition 12.3 in \cite{Milne_2017}.
\end{proof}

\begin{proposition}
  \label{DM_struct}
  The choice of a basis for $M$ determines an isomorphism of $D(M)$
  with a finite product of copies of $\Gm$ and various $\mu_n$.
\end{proposition}
\begin{proof}
  See Proposition 12.3 in \cite{Milne_2017}.
\end{proof}

\begin{lemma}
  \label{grp_like_in_kM}
  \uses{grp_like}
  The group-like elements of $k[M]$ are exactly the image of $M$.
\end{lemma}
\begin{proof}
  \uses{grp_like_li}
  See Lemma 12.4 in \cite{Milne}.
\end{proof}

\begin{definition}
  \label{diag}
  \uses{grp_like}
  An algebraic group $G$ is \emph{diagonalizable}
  if the group-like elements in $\Gamma(G)$ span it as a $k$-vector space.
\end{definition}

\begin{theorem}
  \label{diag_iff_D}
  \uses{grp_alg_sch}
  \uses{diag}
  An algebraic group $G$ is diagonalizable
  if and only if it is isomorphic to $D(M)$ for some commutative group $M$.
\end{theorem}
\begin{proof}
  \uses{grp_like_li}
  See Theorem 12.8 in \cite{Milne_2017}.
\end{proof}

\begin{theorem}
  \label{congr_fggrp_diag}
  \uses{char_lattice}
  \uses{diag}
  The functor $M\rightsquigarrow D(M)$ is a contravariant equivalence
  from the category of finitely generated commutative groups to the category of
  diagonalizable algebraic groups (with quasi-inverse $G \rightsquigarrow X(G)$).
\end{theorem}
\begin{proof}
  \uses{diag_iff_D}
  See Theorem 12.9(a) in \cite{Milne_2017}. Case work required.
\end{proof}
