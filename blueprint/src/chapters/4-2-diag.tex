\section{Diagonalisable groups}


\begin{definition}[The diagonalisable group scheme functor]
  \label{0-diag}
  \uses{0-grp-alg, 0-hopf-alg-equiv-cogrp-alg, 0-spec-hopf}
  \lean{Diag}
  \leanok

  Let $G$ be a commutative group and $S$ a base scheme.
  The diagonalisable group scheme $D_S(G)$ is defined as the base-change of $\Spec \Z[G]$ to $S$.
  For a commutative ring $R$, we write $D_R(G) := D_{\Spec R}(G)$.
\end{definition}


\begin{definition}[Diagonalisable group schemes]
  \label{0-is-diag}
  \uses{0-diag}
  \lean{AlgebraicGeometry.Scheme.IsDiagonalisable}
  \leanok

  An algebraic group $G$ over $\Spec R$ is {\bf diagonalisable} if it is isomorphic to $D_R(G)$ for some commutative group $G$.
\end{definition}


\begin{lemma}[The diagonalisable group scheme torus over $\Spec R$]
  \label{0-diag-spec}
  \uses{0-torus}
  \lean{AlgebraicGeometry.Scheme.diagSpecIso, AlgebraicGeometry.Scheme.splitTorusIso}
  \leanok

  Let $R$ be a commutative ring and $M$ an abelian monoid.
  Then $D_R(M)$ is isomorphic to $\Spec R[M]$.
\end{lemma}
\begin{proof}
  \uses{}
  \leanok

  Ask any toddler on the street.
\end{proof}


\begin{theorem}
  \label{0-diag-iff-grp-like-span}
  \uses{0-diag, 0-is-diag}
  \lean{AlgebraicGeometry.Scheme.isDiagonalisable_iff_span_isGroupLikeElem_eq_top}
  \leanok

  An algebraic group $G$ over a field $k$ is diagonalizable if and only if $\Gamma(G)$ is spanned by its group-like elements.
\end{theorem}
\begin{proof}
  \uses{0-grp-like-lin-indep}
  % \leanok

  See Theorem 12.8 in \cite{Milne_2017}.
\end{proof}


\begin{theorem}
  \label{0-full-faithful-diag}
  \uses{0-diag}
  \lean{}
  \leanok

  Let $R$ be a domain.
  The functor $D_R(G) := G \rightsquigarrow \Spec R[G]$ from the category of groups to the category of group schemes over $\Spec R$ is fully faithful.
\end{theorem}
\begin{proof}
  \uses{0-full-faithful-spec-hopf, 0-full-faithful-grp-alg}
  \leanok

  Compose Propositions \ref{0-full-faithful-spec-hopf} and \ref{0-full-faithful-grp-alg}.

  Also see Theorem 12.9(a) in \cite{Milne_2017}. See SGA III Exposé VIII for a proof that works for $R$ an arbitrary commutative ring.
\end{proof}
