\section{Diagonalisable groups}


\begin{definition}[The diagonalisable group scheme functor]
  \label{0-diag}
  \uses{0-grp-alg, 0-hopf-alg-equiv-cogrp-alg, 0-spec-hopf}
  \lean{Diag}
  \leanok

  Let $G$ be a commutative group and $S$ a base scheme.
  The diagonalisable group scheme $D_S(G)$ is defined as the base-change of $\Spec \Z[G]$ to $S$.
  For a commutative ring $R$, we write $D_R(G) := D_{\Spec R}(G)$.
\end{definition}


\begin{definition}[Diagonalisable group schemes]
  \label{0-is-diag}
  \uses{0-diag}
  \lean{AlgebraicGeometry.Scheme.IsDiagonalisable}
  \leanok

  An algebraic group $G$ over $\Spec R$ is {\bf diagonalisable} if it is isomorphic to $D_R(G)$ for some commutative group $G$.
\end{definition}


\begin{lemma}[The diagonalisable group scheme torus over $\Spec R$]
  \label{0-diag-spec}
  \uses{0-torus}
  \lean{AlgebraicGeometry.Scheme.diagSpecIso, AlgebraicGeometry.Scheme.splitTorusIso}
  \leanok

  Let $R$ be a commutative ring and $M$ an abelian monoid.
  Then $D_R(M)$ is isomorphic to $\Spec R[M]$.
\end{lemma}
\begin{proof}
  \uses{}
  \leanok

  Ask any toddler on the street.
\end{proof}


\begin{theorem}
  \label{0-diag-iff-grp-like-span}
  \uses{0-diag, 0-is-diag}
  \lean{AlgebraicGeometry.Scheme.isDiagonalisable_iff_span_isGroupLikeElem_eq_top}
  \leanok

  An algebraic group $G$ over a field $k$ is diagonalizable if and only if $\Gamma(G)$ is spanned by its group-like elements.
\end{theorem}
\begin{proof}
  \uses{0-grp-like-lin-indep}
  % \leanok

  See Theorem 12.8 in \cite{Milne_2017}.
\end{proof}


\begin{theorem}
  \label{0-full-faithful-diag}
  \uses{0-diag}
  \lean{}
  \leanok

  Let $R$ be a domain.
  The functor $D_R(G) := G \rightsquigarrow \Spec R[G]$ from the category of groups to the category of group schemes over $\Spec R$ is fully faithful.
\end{theorem}
\begin{proof}
  \uses{0-full-faithful-spec-hopf, 0-full-faithful-grp-alg}
  \leanok

  Compose Propositions \ref{0-full-faithful-spec-hopf} and \ref{0-full-faithful-grp-alg}.

  Also see Theorem 12.9(a) in \cite{Milne_2017}. See SGA III Exposé VIII for a proof that works for $R$ an arbitrary commutative ring.
\end{proof}


\begin{proposition}[Morphisms between diagonalisable group schemes are affine]
  \label{0-diag-aff-hom}
  \uses{0-diag}
  % \lean{}
  % \leanok

  Let $S$ be a scheme.
  Let $M, N$ be commutative monoids and $f : M \to N$ a monoid hom.
  Then the map $D_S(f) : D_S(N) \to D_S(M)$ is affine.
\end{proposition}
\begin{proof}
  \uses{}
  % \leanok

  $\Spec f : \Spec \Z[N] \to \Spec \Z[M]$ is affine by definition(?).
  Therefore $D_S(f)$ is affine, as affine morphisms are preserved under base change.
\end{proof}


\begin{proposition}[Closed embeddings between diagonalisable group schemes]
  \label{0-diag-closed-emb}
  \uses{0-diag}
  % \lean{}
  % \leanok

  Let $S$ be a scheme.
  Let $M, N$ be commutative monoids and $f : M \to N$ a surjective monoid hom.
  Then the map $D_S(f) : D_S(N) \to D_S(M)$ is a closed embedding.
\end{proposition}
\begin{proof}
  \uses{}
  % \leanok

  Since $f$ is surjective, the corresponding map $\hat f : \Z[M] \to \Z[N]$ is surjective too.
  Hence $\Spec \hat f : \Spec \Z[N] \to \Spec \Z[M]$ is a closed embedding.
  Therefore $D_S(f)$ is a closed embedding, as closed embeddings are preserved under base change.
\end{proof}


\begin{proposition}[Faithfully flat morphisms between diagonalisable group schemes]
  \label{0-diag-faithful-flat}
  \uses{0-diag}
  % \lean{}
  % \leanok

  Let $S$ be a scheme.
  Let $G, H$ be abelian groups and $f : G \to H$ an injective group hom.
  Then the map $D_S(f) : D_S(H) \to D_S(G)$ is faithfully flat.
\end{proposition}
\begin{proof}
  \uses{0-mon-alg-free}
  % \leanok

  Since $f$ is injective, $\Z[H]$ is a free module over $\Z[G]$ by Proposition \ref{0-mon-alg-free},
  hence the map $\Z[G] \to \Z[H]$ is faithfully flat and so is $\Spec \Z[H] \to \Spec \Z[G]$.
  Therefore $D_S(f)$ is faithfully flat,
  as faithfully flat morphisms are preserved under base change.
\end{proof}


\begin{proposition}[Faithfully flat morphisms between diagonalisable group schemes]
  \label{0-diag-faithful-flat}
  \uses{0-diag}
  % \lean{}
  % \leanok

  Let $S$ be a scheme.
  Let $G, H$ be abelian groups and $f : G \to H$ an injective group hom.
  Then the map $D_S(f) : D_S(H) \to D_S(G)$ is faithfully flat.
\end{proposition}
\begin{proof}
  \uses{0-grp-alg-free}
  % \leanok

  Since $f$ is injective, $\Z[H]$ is a free module over $\Z[G]$ by Proposition \ref{0-grp-alg-free},
  hence the map $\Z[G] \to \Z[H]$ is faithfully flat and so is $\Spec \Z[H] \to \Spec \Z[G]$.
  Therefore $D_S(f)$ is faithfully flat,
  as faithfully flat morphisms are preserved under base change.
\end{proof}


\begin{proposition}[A subgroup of a diagonalisable group scheme is a diagonalisable group scheme]
  \label{0-subgroup-diag}
  \uses{0-diag}
  % \lean{}
  % \leanok

  Let $R$ be a domain. Let $G$ be an abelian group.
  If $H$ is a closed subgroup of $D_R(G)$, then $H$ is a diagonalisable group scheme.
\end{proposition}
\begin{proof}
  \uses{0-is-diag-bialg-quot, 0-ess-image-spec-hopf, 0-full-faithful-spec-hopf, 0-diag-spec}
  % \leanok

  $H$ is a closed subscheme of an affine scheme, hence it is affine.
  By Proposition \ref{0-ess-image-spec-hopf}, write $H = \Spec A$ where $A$ is a $R$-Hopf algebra.
  The closed subgroup embedding $H \hookrightarrow D_R(G)$
  becomes a surjective bialgebra hom $R[G] \to A$ by
  Propositions \ref{0-diag-spec} and \ref{0-full-faithful-spec-hopf}.
  By Proposition \ref{0-is-diag-bialg-quot},
  $A$ is a diagonalisable bialgebra and therefore $H$ is a diagonalisable group scheme.
\end{proof}


\begin{proposition}[Diagonalisable group scheme of a torsion group is disconnected]
  \label{0-diag-tors}
  \uses{0-diag}
  % \lean{}
  % \leanok

  Let $G$ be an abelian group with an element of torsion $n$.
  Let $R$ be a commutative ring with $n$ invertible.
  Then $D_R(G)$ is disconnected.
\end{proposition}
\begin{proof}
  \uses{0-diag-spec}
  % \leanok

  Say $x \in G$ is such that $x^n = 1$. Then
  \[e : R[G] := \frac 1n \sum_{i = 0}^n x^i\]
  is such that $e ^ 2 = e$.
  We are done by Proposition \ref{0-diag-spec}.
\end{proof}
