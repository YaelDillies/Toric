\section{Over category}


\begin{proposition}[Sliced adjoint functors]
  \label{0-slice-adj}
  \uses{}
  \lean{CategoryTheory.Over.postAdjunctionRight}
  \leanok
  \mathlibok

  If $a : F \vdash G$ is an adjunction between $F : C \to D$ and $G : D \to C$ and $X : C$, then there is an adjunction between $F / X : C / X \to D / F(X)$ and $G / X : D / F(X) \to C / X$.
\end{proposition}
\begin{proof}
  \uses{}
  \leanok

  See https://ncatlab.org/nlab/show/sliced+adjoint+functors+--+section.
\end{proof}


\begin{proposition}[Limit-preserving functors lift to over categories]
  \label{0-over-lim}
  \uses{}
  \lean{CategoryTheory.Limits.PreservesLimitsOfShape.overPost, CategoryTheory.Limits.PreservesLimitsOfSize.overPost, CategoryTheory.Limits.PreservesFiniteProducts.overPost}
  \leanok

  Let $J$ be a shape (i.e. a category). Let $\widetilde J$ denote the category which is the same as $J$, but has an extra object $*$ which is terminal.
  If $F : C \to D$ is a functor preserving limits of shape $\widetilde J$, then the obvious functor $C / X \to D / F(X)$ preserves limits of shape $J$.
\end{proposition}
\begin{proof}
  \uses{}
  \leanok

  Extend a functor $K\colon  J \to C / X$ to a functor $\widetilde K\colon \widetilde J \to C$, by letting $\widetilde K (*) = X$.
\end{proof}


\begin{proposition}[Essential image of a sliced functor]
  \label{0-ess-image-over}
  \uses{}
  \lean{CategoryTheory.Functor.essImage_overPost}
  \leanok

  If $F : C \to D$ is a full functor between cartesian-monoidal categories, then $F / X : C / X \hom D / F(X)$ has the same essential image as $F$.
\end{proposition}
\begin{proof}
  \uses{}
  \leanok

  Transfer all diagrams.
\end{proof}
