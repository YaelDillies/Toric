\section{Affine Monoids}


\begin{lemma}[Multivariate Laurent polynomials are an integral domain]
  \label{0-mv-laurent-poly-domain}
  \uses{}
  \lean{AddMonoidAlgebra.instIsDomain}
  \leanok

  Multivariate Laurent polynomials over an integral domain are an integral domain.
\end{lemma}
\begin{proof}
  \uses{}
  \leanok

  Come on.
\end{proof}


\begin{definition}[Affine monoid]
  \label{0-aff-mon}
  \lean{AddCancelCommMonoid, AddMonoid.FG, AddMonoid.IsTorsionFree}
  \leanok

  An \emph{affine monoid} is a finitely generated commutative monoid which is:
  \begin{itemize}
    \item cancellative: if $a + c = b + c$ then $a = b$, and
    \item torsion-free: if $n a = n b$ then $a = b$ (for $n \geq 1$).
  \end{itemize}
\end{definition}


\begin{proposition}[Embedding an affine monoid inside a lattice]
  \label{0-embed-aff-mon}
  \uses{0-aff-mon}
  \lean{AffineMonoid.embedding, AffineMonoid.embedding_injective}
  \leanok

  If $M$ is an affine monoid, then $M$ can be embedded inside $\Z^n$ for some $n$.
\end{proposition}
\begin{proof}
  \uses{}
  \leanok

  Embed $M$ inside its Grothendieck group $G$. Prove that $G$ is finitely generated free.
\end{proof}


\begin{proposition}[Affine monoid algebras are domains]
  \label{0-aff-mon-alg-domain}
  \uses{0-aff-mon}
  \lean{MonoidAlgebra.finiteType_of_fg}
  \leanok

  If $R$ is an integral domain $M$ is an affine monoid, then $R[M]$ is an integral domain and is a finitely generated $R$-algebra.
\end{proposition}
\begin{proof}
  \uses{0-mv-laurent-poly-domain, 0-embed-aff-mon}
  \leanok

  $i : R[M] \hookrightarrow R[\Z M]$ injects into an integral domain so is an integral domain. It's finitely generated by $\chi^{a_i}$ where $\mathcal A = \{a_1, \dotsc, a_s\}$ is a finite generating set for $M$.
\end{proof}


\begin{definition}[Irreducible element]
  \label{0-irred}
  \uses{}
  \lean{Irreducible}
  \leanok

  An element $x$ of a monoid $M$ is \emph{irreducible} if $x = y + z$ implies $y = 0$ or $z = 0$.
\end{definition}


\begin{proposition}[Irreducible elements lie in all sets generating a salient monoid]
  \label{0-irred-subset-gen}
  \uses{0-irred}
  \lean{addIrreducible_subset_of_addSubmonoidClosure_eq_top}
  \leanok

  If $M$ is a monoid with a single unit, and $S$ is a set generating $M$, then $S$ contains all irreducible elements of $M$.
\end{proposition}
\begin{proof}
  \uses{}
  \leanok

  Assume $p$ is an irreducible element. Since $S$ generates $M$, write
  \[
    p = \sum_i a_i
  \]
  where the $a_i$ are finitely many elements (not necessarily distinct) elements of $S$. Since $p$ is irreducible, we must have
  \[
    p = a_i \in S
  \]
  for some $i$.
\end{proof}


\begin{proposition}[A salient finitely generated monoid has finitely many irreducible elements]
  \label{0-irred-finite}
  \uses{0-irred}
  \lean{finite_addIrreducible}
  \leanok

  If $M$ is a finitely generated monoid with a single unit, then only finitely many elements of $M$ are irreducible.
\end{proposition}
\begin{proof}
  \uses{0-irred-subset-gen}
  \leanok

  Let $S$ be a finite set generating $M$. Write $I$ the set of irreducible elements. By Proposition \ref{0-irred-subset-gen}, $I \subseteq S$. Hence $I$ is finite.
\end{proof}


\begin{proposition}[A salient finitely generated cancellative monoid is generated by its irreducible elements]
  \label{0-irred-gen}
  \uses{0-irred}
  \lean{AddSubmonoid.closure_addIrreducible}
  \leanok

  If $M$ is a finitely generated cancellative monoid with a single unit, then $M$ is generated by its irreducible elements.
\end{proposition}
\begin{proof}
  \uses{}
  \leanok

  We do not follow the proof from \cite{Cox_2011}.

  Let $S$ be a finite minimal generating set and assume for contradiction that $r \in S$ is reducible, say $r = a + b$ where $a, b$ are non-units. Write
  \[a = \sum_{s \in S} m_s s, b = \sum_{s \in S} n_s s\]
  for some $m_s, n_s \in \N$, so that
  \[r = \sum_{s \in S} (m_s + n_s) s.\]
  We distinguish three cases
  \begin{itemize}
    \item $m_r + n_r = 0$. Then
    \[r = \sum_{s \in S \setminus \{r\}} (m_s + n_s) s \in \langle S \setminus \{r\}\rangle\]
    contradicting the minimality of $S$.
    \item $m_r + n_r = 1$. Then
    \begin{align*}
      & 0 = \sum_{s \in S \setminus \{r\}} (m_s + n_s) s
      & \implies \forall s \in S \setminus \{r\}, m_s s = n_s s = 0
    \end{align*}
    Furthermore, either $m_r = 0$ or $n_r = 0$, so $a = 0$ or $b = 0$, contradicting the fact that $a$ and $b$ are non-units.
    \item $m_r + n_r \ge 2$. Then
    \[0 = r + \sum_{s \in S \setminus \{r\}} (m_s + n_s) s\]
    and $r = 0$, contradicting the minimality of $S$ once again.
  \end{itemize}
\end{proof}
