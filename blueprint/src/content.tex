% In this file you should put the actual content of the blueprint.
% It will be used both by the web and the print version.
% It should *not* include the \begin{document}
%
% If you want to split the blueprint content into several files then
% the current file can be a simple sequence of \input. Otherwise It
% can start with a \section or \chapter for instance.

\setcounter{chapter}{-1} % Make Prerequisites be Chapter 0

\chapter{Prerequisites}

\section{Affine Monoids}


\begin{lemma}[Multivariate Laurent polynomials are an integral domain]
  \label{0-mv-laurent-poly-domain}
  \uses{}
  \lean{AddMonoidAlgebra.instIsDomain}
  \leanok

  Multivariate Laurent polynomials over an integral domain are an integral domain.
\end{lemma}
\begin{proof}
  \uses{}
  \leanok

  Come on.
\end{proof}


\begin{definition}[Affine monoid]
  \label{0-aff-mon}
  \lean{AddCancelCommMonoid, AddMonoid.FG, AddMonoid.IsTorsionFree}
  \leanok

  An \emph{affine monoid} is a finitely generated commutative monoid which is:
  \begin{itemize}
    \item cancellative: if $a + c = b + c$ then $a = b$, and
    \item torsion-free: if $n a = n b$ then $a = b$ (for $n \geq 1$).
  \end{itemize}
\end{definition}


\begin{proposition}[Embedding an affine monoid inside a lattice]
  \label{0-embed-aff-mon}
  \uses{0-aff-mon}
  \lean{AffineMonoid.embedding, AffineMonoid.embedding_injective}
  \leanok

  If $M$ is an affine monoid, then $M$ can be embedded inside $\Z^n$ for some $n$.
\end{proposition}
\begin{proof}
  \uses{}
  \leanok

  Embed $M$ inside its Grothendieck group $G$. Prove that $G$ is finitely generated free.
\end{proof}


\begin{proposition}[Affine monoid algebras are domains]
  \label{0-aff-mon-alg-domain}
  \uses{0-aff-mon}
  \lean{MonoidAlgebra.finiteType_of_fg}
  \leanok

  If $R$ is an integral domain $M$ is an affine monoid, then $R[M]$ is an integral domain and is a finitely generated $R$-algebra.
\end{proposition}
\begin{proof}
  \uses{0-mv-laurent-poly-domain, 0-embed-aff-mon}
  \leanok

  $i : R[M] \hookrightarrow R[\Z M]$ injects into an integral domain so is an integral domain. It's finitely generated by $\chi^{a_i}$ where $\mathcal A = \{a_1, \dotsc, a_s\}$ is a finite generating set for $M$.
\end{proof}


\begin{definition}[Irreducible element]
  \label{0-irred}
  \uses{}
  \lean{Irreducible}
  \leanok

  An element $x$ of a monoid $M$ is \emph{irreducible} if $x = y + z$ implies $y = 0$ or $z = 0$.
\end{definition}


\begin{proposition}[Irreducible elements lie in all sets generating a salient monoid]
  \label{0-irred-subset-gen}
  \uses{0-irred}
  \lean{addIrreducible_subset_of_addSubmonoidClosure_eq_top}
  \leanok

  If $M$ is a monoid with a single unit, and $S$ is a set generating $M$, then $S$ contains all irreducible elements of $M$.
\end{proposition}
\begin{proof}
  \uses{}
  \leanok

  Assume $p$ is an irreducible element. Since $S$ generates $M$, write
  \[
    p = \sum_i a_i
  \]
  where the $a_i$ are finitely many elements (not necessarily distinct) elements of $S$. Since $p$ is irreducible, we must have
  \[
    p = a_i \in S
  \]
  for some $i$.
\end{proof}


\begin{proposition}[A salient finitely generated monoid has finitely many irreducible elements]
  \label{0-irred-finite}
  \uses{0-irred}
  \lean{finite_addIrreducible}
  \leanok

  If $M$ is a finitely generated monoid with a single unit, then only finitely many elements of $M$ are irreducible.
\end{proposition}
\begin{proof}
  \uses{0-irred-subset-gen}
  \leanok

  Let $S$ be a finite set generating $M$. Write $I$ the set of irreducible elements. By Proposition \ref{0-irred-subset-gen}, $I \subseteq S$. Hence $I$ is finite.
\end{proof}


\begin{proposition}[A salient finitely generated cancellative monoid is generated by its irreducible elements]
  \label{0-irred-gen}
  \uses{0-irred}
  \lean{AddSubmonoid.closure_addIrreducible}
  \leanok

  If $M$ is a finitely generated cancellative monoid with a single unit, then $M$ is generated by its irreducible elements.
\end{proposition}
\begin{proof}
  \uses{}
  \leanok

  We do not follow the proof from \cite{Cox_2011}.

  Let $S$ be a finite minimal generating set and assume for contradiction that $r \in S$ is reducible, say $r = a + b$ where $a, b$ are non-units. Write
  \[a = \sum_{s \in S} m_s s, b = \sum_{s \in S} n_s s\]
  for some $m_s, n_s \in \N$, so that
  \[r = \sum_{s \in S} (m_s + n_s) s.\]
  We distinguish three cases
  \begin{itemize}
    \item $m_r + n_r = 0$. Then
    \[r = \sum_{s \in S \setminus \{r\}} (m_s + n_s) s \in \langle S \setminus \{r\}\rangle\]
    contradicting the minimality of $S$.
    \item $m_r + n_r = 1$. Then
    \begin{align*}
      & 0 = \sum_{s \in S \setminus \{r\}} (m_s + n_s) s
      & \implies \forall s \in S \setminus \{r\}, m_s s = n_s s = 0
    \end{align*}
    Furthermore, either $m_r = 0$ or $n_r = 0$, so $a = 0$ or $b = 0$, contradicting the fact that $a$ and $b$ are non-units.
    \item $m_r + n_r \ge 2$. Then
    \[0 = r + \sum_{s \in S \setminus \{r\}} (m_s + n_s) s\]
    and $r = 0$, contradicting the minimality of $S$ once again.
  \end{itemize}
\end{proof}

\section{Tensor Product}


\begin{lemma}[The tensor product of linearly independent families]
  \label{0-tensor-lin-indep}
  \uses{}
  \lean{LinearIndependent.tmul_of_isDomain}
  \leanok

  Let $R$ be a domain and $M, N$ two $R$-semimodules.
  If $f$ and $g$ are linearly independent families of points in $M$ and $N$, then $(i, j) \mapsto f i \ox g j$ is a linearly independent family of points in $M \ox N$.
\end{lemma}
\begin{proof}
  \uses{}
  \leanok

  TODO(Andrew)
\end{proof}

\section{Hopf algebras}


\subsection{Group-like elements}


\begin{definition}[Group-like elements]
  \label{0-grp-like}
  \uses{}
  \lean{Bialgebra.IsGroupLikeElem}
  \leanok

  An element $a$ of a bi-algebra $A$ is \emph{group-like} if it is a unit and $\Delta(a) = a \ox a$, where $\Delta$ is the comultiplication map.
\end{definition}


\begin{proposition}[Group-like elements form a group]
  \label{0-grp-like-grp}
  \uses{0-grp-like}
  \lean{Bialgebra.GroupLike.instGroup, Bialgebra.GroupLike.instCommGroup}
  \leanok

  Group-like elements of a bi-algebra $A$ form a group under multiplication.
\end{proposition}
\begin{proof}
  \uses{}
  \leanok

  Check that group-like elements are closed under unit, multiplication and inverses.
\end{proof}


\begin{lemma}[Bialgebra homs preserve group-like elements]
  \label{0-grp-like-map}
  \uses{0-grp-like}
  \lean{Bialgebra.IsGroupLikeElem.map}
  \leanok

  Let $f : A \to B$ be a bi-algebra hom. If $a \in A$ is group-like, then $f(a)$ is group-like too.
\end{lemma}
\begin{proof}
  \uses{}
  \leanok

  $a$ is a unit, so $f(a)$ is a unit too. Then
  \[
    f(a) \ox f(a) = (f \ox f)(\Delta_A(a)) = \Delta_B(f(a))
  \]
  so $f(a)$ is group-like.
\end{proof}


\begin{lemma}[Independence of group-like elements]
  \label{0-grp-like-lin-indep}
  \uses{0-grp-like}
  \lean{Coalgebra.linearIndepOn_isGroupLikeElem}
  \leanok

  The group-like elements in $A$ are linearly independent.
\end{lemma}
\begin{proof}
  \uses{0-tensor-lin-indep}
  \leanok

  See Lemma 4.23 in \cite{Milne_2017}.
\end{proof}


\begin{lemma}[Group-like elements in a group algebra]
  \label{0-grp-like-grp-alg}
  \uses{0-grp-like}
  \lean{MonoidAlgebra.isGroupLikeElem_iff_mem_range_of}
  \leanok

  The group-like elements of $k[M]$ are exactly the image of $M$.
\end{lemma}
\begin{proof}
  \uses{0-grp-like-lin-indep}
  \leanok

  See Lemma 12.4 in \cite{Milne_2017}.
\end{proof}


\subsection{The group algebra functor}


\begin{proposition}[The antipode is a antihomomorphism]
  \label{0-hopf-antipode-antihom}
  \uses{}
  \lean{HopfAlgebra.antipode_mul_antidistrib, HopfAlgebra.antipode_mul_distrib}
  \leanok

  If $A$ is a $R$-Hopf algebra, then the antipode map $s : A \to A$ is anti-commutative, ie $s(a * b) = s(b) * s(a)$. If further $A$ is commutative, then $s(a * b) = s(a) * s(b)$.
\end{proposition}
\begin{proof}
  \uses{}
  % \leanok

  Any standard reference will have a proof.
\end{proof}


\begin{proposition}[Hopf algebras are cogroup objects in the category of algebras]
  \label{0-hopf-cogrp-alg}
  \uses{}
  \lean{grp_Class_op_commAlgOf, isMon_Hom_commAlgOfHom, bialgebra_unop, hopfAlgebra_unop, IsMon_Hom.toBialgHom}
  \leanok

  From a $R$-Hopf algebra, one can build a cogroup object in the category of $R$-algebras.

  From a cogroup object in the category of $R$-algebras, one can build a $R$-Hopf algebra.
\end{proposition}
\begin{proof}
  \uses{0-hopf-antipode-antihom}
  \leanok

  Turn the arrows around.
\end{proof}


\begin{definition}[The group algebra functor]
  \label{0-grp-alg}
  \uses{0-hopf-cogrp-alg}
  \lean{commGrpAlgebra}
  \leanok

  For a commutative ring $R$, we have a functor $G \rightsquigarrow R[G] : \Grp \to \Hopf_R$.
\end{definition}


\begin{proposition}[The group algebra functor is fully faithful]
  \label{0-full-faithful-grp-alg}
  \uses{0-grp-alg}
  \lean{MonoidAlgebra.mapDomainBialgHomMulEquiv, AddMonoidAlgebra.mapDomainBialgHomMulEquiv, commGrpAlgebra.fullyFaithful, commGrpAlgebra.instFull, commGrpAlgebra.instFaithful}
  \leanok

  For a field $K$, the functor $G \rightsquigarrow K[G]$ from the category of groups to the category of Hopf algebras over $K$ is fully faithful.
\end{proposition}
\begin{proof}
  \uses{0-grp-like-grp-alg, 0-grp-like-map}
  \leanok

  It is clearly faithful.
  Now for the full part, if $f : K[G] \to K[H]$ is a Hopf algebra hom, then we get a series of maps
  \[
    G \simeq \text{ group-like elements of } R[G] \to \text{ group-like elements of } R[H] \simeq H
  \]
  and each map separately is clearly multiplicative.
\end{proof}

\section{Group Schemes}


\subsection{Correspondence between Hopf algebras and affine group schemes}


We want to show that Hopf algebras correspond to affine group schemes.
This can easily be done categorically assuming both categories on either side are defined thoughtfully.
However, the categorical version will not be workable with if we do not also have links to the non-categorical notions.
Therefore, one solution would be to build the left, top and right edges of the following
diagram so that the bottom edge can be obtained by composing the three:
%   Cogrp Mod_R   ⥤        Grp Sch_{Spec R}
%       ↑ ↓                        ↓
% R-Hopf algebras → Affine group schemes over Spec R


\subsubsection{Bundling/unbundling Hopf algebras}


We have already done the left edge in the previous section.


\subsubsection{Spec of a Hopf algebra}


Now let's do the top edge.


\begin{proposition}[Sliced adjoint functors]
  \label{0-slice-adj}
  \uses{}
  % \lean{}
  % \leanok

  If $a : F \vdash G$ is an adjunction between $F : C \to D$ and $G : D \to C$ and $X : C$, then there is an adjunction between $F / X : C / X \to D / F(X)$ and $G / X : D / F(X) \to C / X$.
\end{proposition}
\begin{proof}
  \uses{}
  % \leanok

  See https://ncatlab.org/nlab/show/sliced+adjoint+functors+--+section.
\end{proof}


\begin{proposition}[Limit-preserving functors lift to over categories]
  \label{0-over-lim}
  \uses{}
  \lean{CategoryTheory.Limits.PreservesLimitsOfShape.overPost, CategoryTheory.Limits.PreservesLimitsOfSize.overPost, CategoryTheory.Limits.PreservesFiniteProducts.overPost}
  \leanok

  Let $J$ be a shape (i.e. a category). Let $\widetilde J$ denote the category which is the same as $J$, but has an extra object $*$ which is terminal.
  If $F : C \to D$ is a functor preserving limits of shape $\widetilde J$, then the obvious functor $C / X \to D / F(X)$ preserves limits of shape $J$.
\end{proposition}
\begin{proof}
  \uses{}
  \leanok

  Extend a functor $K\colon  J \to C / X$ to a functor $\widetilde K\colon \widetilde J \to C$, by letting $\widetilde K (*) = X$.
\end{proof}


\begin{proposition}[Fully faithful product-preserving functors lift to group objects]
  \label{0-full-faithful-grp}
  \uses{}
  \lean{CategoryTheory.Functor.Faithful.mapGrp, CategoryTheory.Functor.Full.mapGrp}
  \leanok

  If a finite-products-preserving functor $F : C \to D$ is fully faithful, then so is $\Grp(F) : \Grp C \to \Grp D$.
\end{proposition}
\begin{proof}
  \uses{}
  \leanok

  Faithfulness is immediate.

  For fullness, assume $f : F(G) \to F(H)$ is a morphism. By fullness of $F$, find $g : G \to H$ such that $F(g) = f$. $g$ is a morphism because we can pull back each diagram from $D$ to $C$ along $F$ which is faithful.
\end{proof}


\begin{definition}[Spec as a functor on algebras]
  \label{0-spec-alg}
  \uses{}
  \lean{algSpec}
  \leanok

  Spec is a contravariant functor from the category of $R$-algebras to the category of schemes over $\Spec_R$.
\end{definition}


\begin{proposition}[Spec as a functor on algebras is fully faithful]
  \label{0-full-faithful-spec-alg}
  \uses{0-spec-alg}
  \lean{algSpec.instPreservesLimits, algSpec.instFull, algSpec.instFaithful, algSpec.fullyFaithful}
  \leanok

  Spec is a fully faithful contravariant functor from the category of $R$-algebras to the category of schemes over $\Spec_R$, preserving all limits.
\end{proposition}
\begin{proof}
  \uses{0-slice-adj, 0-over-lim}
  \leanok

  $\Spec : \Ring \to \Sch$ is a fully faithful contravariant functor which preserves all limits, hence so is $\Spec : \Ring_R \to \AffSch_{\Spec R}$ by Proposition \ref{0-over-lim} (alternatively, by Proposition \ref{0-slice-adj}).
\end{proof}


\begin{definition}[Spec as a functor on Hopf algebras]
  \label{0-spec-hopf}
  \uses{0-full-faithful-spec-alg}
  \lean{hopfSpec}
  \leanok

  Spec is a fully faithful contravariant functor from the category of $R$-algebras to the category of group schemes over $\Spec_R$.
\end{definition}


\begin{proposition}[Spec as a functor on Hopf algebras is fully faithful]
  \label{0-full-faithful-spec-hopf}
  \uses{0-spec-alg}
  \lean{hopfSpec.instFull, hopfSpec.instFaithful, hopfSpec.fullyFaithful}
  \leanok

  Spec is a fully faithful contravariant functor from the category of $R$-Hopf algebras to the category of group schemes over $\Spec_R$.
\end{proposition}
\begin{proof}
  \uses{0-full-faithful-grp, 0-full-faithful-spec-alg}
  \leanok

  $\Spec : \Ring_R \to \Sch_{\Spec R}$ is a fully faithful contravariant functor preserving all limits according to Proposition \ref{0-spec-alg}, therefore $\Spec : \Hopf_R \to \GrpSch_{\Spec R}$ too is fully faithful according to \ref{0-full-faithful-grp}.
\end{proof}


\subsection{Essential image of Spec on Hopf algebras}


Finally, let's do the right edge.


\begin{proposition}[Essential image of a sliced functor]
  \label{0-ess-image-over}
  \uses{}
  \lean{CategoryTheory.Functor.essImage_overPost}
  \leanok

  If $F : C \to D$ is a full functor between cartesian-monoidal categories, then $F / X : C / X \hom D / F(X)$ has the same essential image as $F$.
\end{proposition}
\begin{proof}
  \uses{}
  \leanok

  Transfer all diagrams.
\end{proof}


\begin{proposition}[Equivalences lift to group object categories]
  \label{0-grp-equiv}
  \uses{}
  \lean{CategoryTheory.Equivalence.mapGrp}
  \leanok

  If $e : C \backsimeq D$ is an equivalence of cartesian-monoidal categories, then $\Grp(e) : \Grp(C) \backsimeq \Grp(D)$ too is an equivalence of categories.
\end{proposition}
\begin{proof}
  \uses{}
  \leanok

  Transfer all diagrams.
\end{proof}


\begin{proposition}[Essential image of a functor on group objects]
  \label{0-ess-image-grp}
  \uses{}
  \lean{CategoryTheory.Functor.essImage_mapGrp}
  \leanok

  If $F : C \to D$ is a fully faithful functor between cartesian-monoidal categories, then $\Grp(F) : \Grp(C) \hom \Grp(D)$ has the same essential image as $F$.
\end{proposition}
\begin{proof}
  \uses{0-grp-equiv}
  % \leanok

  Transfer all diagrams.
\end{proof}


\begin{proposition}[Essential image of Spec on algebras]
  \label{0-ess-image-spec-alg}
  \uses{0-spec-alg}
  \lean{essImage_algSpec}
  \leanok

  The essential image of $\Spec : \Ring_R \to \Sch_{\Spec R}$ is precisely affine schemes over $\Spec R$.
\end{proposition}
\begin{proof}
  \uses{0-ess-image-over}
  \leanok

  Direct consequence of Proposition \ref{0-ess-image-over}.
\end{proof}


\begin{proposition}[Essential image of Spec on Hopf algebras]
  \label{0-ess-image-spec-hopf}
  \uses{0-spec-hopf}
  \lean{essImage_hopfSpec}
  \leanok

  The essential image of $\Spec : \Hopf_R \to \GrpSch_{\Spec R}$ is precisely affine group schemes over $\Spec R$.
\end{proposition}
\begin{proof}
  \uses{0-ess-image-grp, 0-ess-image-spec-alg, 0-full-faithful-spec-alg}
  \leanok

  Direct consequence of Propositions \ref{0-ess-image-grp} and \ref{0-ess-image-spec-alg}.
\end{proof}


\subsection{Diagonalisable groups}


\begin{definition}
  \label{0-spec-grp-alg}
  \uses{0-spec-hopf}
  \lean{hopfSpec, MonoidAlgebra}
  \leanok

  For a commutative group $G$ we define $D_R(G)$ as the spectrum $\Spec R[G]$ of the group algebra $R[G]$.
\end{definition}


\begin{definition}
  \label{0-diag}
  \uses{0-spec-grp-alg}
  \lean{AlgebraicGeometry.Scheme.IsDiagonalisable}
  % \leanok

  An algebraic group $G$ over $\Spec R$ is {\bf diagonalisable} if it is isomorphic to $D_R(G)$ for some commutative group $G$.
\end{definition}


\begin{theorem}
  \label{0-diag-iff-grp-like-span}
  \uses{0-spec-grp-alg, 0-diag}
  \lean{AlgebraicGeometry.Scheme.isDiagonalisable_iff_span_isGroupLikeElem_eq_top}
  \leanok

  An algebraic group $G$ over a field $k$ is diagonalizable if and only if group-like elements span $\Gamma(G)$.
\end{theorem}
\begin{proof}
  \uses{0-grp-like-lin-indep}
  % \leanok

  See Theorem 12.8 in \cite{Milne_2017}.
\end{proof}


\begin{theorem}
  \label{0-full-faithful-spec-grp-alg}
  \uses{0-spec-grp-alg}
  \lean{specCommGrpAlgebra.fullyFaithful, specCommGrpAlgebra.instFull, specCommGrpAlgebra.instFaithful}
  \leanok

  For a field $k$, $D_k$ is a fully faithful contravariant functor from the category of commutative groups to the category of group schemes over $\Spec k$.
\end{theorem}
\begin{proof}
  \uses{0-full-faithful-spec-hopf, 0-full-faithful-grp-alg}
  \leanok

  Compose Propositions \ref{0-full-faithful-spec-hopf} and \ref{0-full-faithful-grp-alg}.

  Also see Theorem 12.9(a) in \cite{Milne_2017}. See SGA III Exposé VIII for a proof that works for $R$ an arbitrary commutative ring in place of $k$.
\end{proof}


\chapter{Affine Toric Varieties}

\section{Introduction to Affine Toric Varieties}


\subsection{The Torus}


\begin{definition}[The torus]
  \label{1-1-torus}
  \lean{AlgebraicGeometry.Scheme.SplitTorus, AlgebraicGeometry.Scheme.SplitTorus.instCanonicallyOver}
  \leanok

  The split torus $\Gm^n$ over a scheme $S$ is the pullback of $\Spec \Z[x_1^{\pm 1}, \dotsc, x_n^{\pm 1}]$ along the unique map $S \to \Spec \Z$.
\end{definition}


\begin{lemma}[The torus over $\Spec R$]
  \label{1-1-torus-spec}
  \uses{1-1-torus}
  \lean{AlgebraicGeometry.Scheme.splitTorusIsoSpec, AlgebraicGeometry.Scheme.splitTorusIsoSpecOver}
  \leanok

  The split torus over $\Spec R$ is isomorphic to $\Spec(R[x_1^{\pm 1}, \dotsc, x_n^{\pm 1}])$.
\end{lemma}
\begin{proof}
  \uses{}
  % \leanok

  Ask any toddler on the street.
\end{proof}


\begin{definition}
  \label{1-1-cochar}
  \uses{1-1-torus}
  \lean{AlgebraicGeometry.Scheme.Over.cochar}
  \leanok

  One-parameter subgroup: a group morphism $\lambda : \Gm \to \Gm^n$.
\end{definition}


\begin{definition}
  \label{1-1-char}
  \uses{1-1-torus}
  \lean{AlgebraicGeometry.Scheme.Over.char}
  \leanok

  The {\bf character lattice} $M = X(\Gm^n) := \Hom_{\mathsf{GrpSch}}(\Gm^n, \Gm)$.
  An element is (unsurprisingly) called a {\bf character}.
\end{definition}


\begin{proposition}
  \label{1-1-char-torus}
  \uses{1-1-char}
  % \lean{}
  % \leanok

  $M = X(\Gm^n) \cong \Z^n$. For $m \in M$ we write $\chi^m$ for the corresponding character.
\end{proposition}
\begin{proof}
  \uses{0-full-faithful-spec-grp-alg, 1-1-torus-spec}
  % \leanok

\end{proof}


\begin{definition}
  \label{1-1-cochar-lat}
  \uses{1-1-cochar}
  % \lean{}
  % \leanok

  One-parameter subgroup/cocharacter lattice $N := \Hom(\Gm, \Gm^n)$.
\end{definition}


\begin{proposition}[Proposition 1.1.1(a)]
  \label{1-1-1-group-hom-subtorus}
  \uses{1-1-torus}
  % \lean{}
  % \leanok

  If $T_1, T_2$ are tori and $\Phi : T_1 \to T_2$ is a morphism which is a group homomorphism, then $\im \Phi$ is a closed subvariety which is a torus.
\end{proposition}
\begin{proof}
  \uses{0-full-faithful-spec-grp-alg, 1-1-torus-spec}
  % \leanok


\end{proof}


\begin{proposition}[A subgroup of a torus is a torus]
  \label{1-1-1-subgroup-subtorus}
  \uses{1-1-torus}
  % \lean{}
  % \leanok

  If $H \subseteq T$ is an irreducible subvariety which is a subgroup, then $H$ is a torus.
\end{proposition}
\begin{proof}
  \uses{0-full-faithful-spec-grp-alg, 1-1-torus-spec}
  % \leanok

  TODO
\end{proof}


\begin{definition}[The character eigenspace]
  \label{1-1-char-eigenspace}
  \uses{1-1-char}
  % \lean{}
  % \leanok

  For a finite dimensional representation of a torus $T$ on $W$, the {\bf character eigenspace} of a character $\chi \in X(T)$ is
  \[
    W_m = \{w\in W : t\cdot w = \chi(t)\text{ for all } t\in T \}.
  \]
\end{definition}


\begin{proposition}[Decomposition into character eigenspaces]
  \label{1-1-2-char-eigenspace-direct-sum}
  \uses{1-1-char-eigenspace}

  The space decomposes into the direct sum of the character eigenspaces.
\end{proposition}
\begin{proof}
  \uses{}
  % \leanok

  TODO
\end{proof}


\begin{definition}
  \label{1-1-char-cochar-pairing}
  \uses{1-1-char}
  \uses{1-1-cochar-lat}
  % \lean{}
  % \leanok

  Character lattice and one-parameter subgroup pairing.
\end{definition}


\begin{proposition}
  \label{1-1-cochar-torus}
  \uses{1-1-cochar-lat}
  % \lean{}
  % \leanok

  $N = \Hom(M, \Z) \cong \Z^n$. For $u \in N$ we write $\lambda^u$ for the corresponding cocharacter.
\end{proposition}
\begin{proof}
  \uses{1-1-char-torus, 1-1-char-cochar-pairing}
  % \leanok

\end{proof}


\subsection{The Definition of Affine Toric Variety}


\begin{definition}
  \label{1-1-3-aff-tor-var}
  \uses{1-1-torus}
  \lean{AlgebraicGeometry.ToricVariety}
  \leanok

  A {\bf toric variety} is a variety $X$ with
  \begin{itemize}
    \item an open embedding $T := (\bbC^\times)^n \hookrightarrow X$ with dense image
    \item such that the natural action $T \times T \to T$ of the torus on itself extends to an (algebraic) action $T \times X \to X$.
  \end{itemize}
\end{definition}


\subsection{Lattice Points}


\begin{definition}
  \label{1-1-phiA}
  \uses{1-1-char}
  % \lean{}
  % \leanok

  Given a finite set $\MCA = \{a_1, \dotsc, a_s\} \subseteq M$, define $\Phi_\MCA : T \to \mathbb{A}^s$ given by $\Phi_{\mathcal A} (t) = (\chi^{a_1} (t), \dotsc, \chi^{a_s} (t))$.
\end{definition}


\begin{definition}
  \label{1-1-7-ya}
  \uses{1-1-phiA}
  % \lean{}
  % \leanok

  $Y_\MCA$ is the (Zariski) closure of $\im \Phi_\MCA$ in $\mathbb A^s$.
\end{definition}


\begin{proposition}
  \label{1-1-8-aff-tor-var-ya}
  \uses{1-1-7-ya}
  \uses{1-1-char}

  Proposition 1.1.8
\end{proposition}
\begin{proof}
  \uses{1-1-1-group-hom-subtorus}
  % \leanok

  TODO
\end{proof}


\subsection{Toric Ideals}


\begin{proposition}
  \label{1-1-9-ideal-ya}
  \uses{1-1-7-ya}
  % \lean{}
  % \leanok

  The ideal of the affine toric variety $Y_\MCA$ is
  \[
    I(Y_\MCA) = \langle x^{\ell_+} - x^{\ell_-} | \ell \in L\rangle
  \]
\end{proposition}
\begin{proof}
  \uses{}
  % \leanok

  See \cite{Cox_2011}.
\end{proof}


\begin{definition}
  \label{1-1-10-lattice-ideal}
  \uses{}
  \lean{AddMonoidAlgebra.monoidIdeal}
  \leanok

  The ideal $I_L = \langle x^\alpha - x^\beta | \alpha, \beta \in \N^s \text{ and } \alpha - \beta \in L\rangle$ is called the {\bf lattice ideal} of the lattice $L \subseteq \Z^s$.

  A toric ideal is a prime lattice ideal.
\end{definition}


\begin{definition}
  \label{1-1-10-toric-ideal}
  \uses{1-1-10-lattice-ideal}
  \lean{AddMonoidAlgebra.IsToricIdeal}
  \leanok
  A {\bf toric ideal} is a prime lattice ideal.
\end{definition}


\begin{proposition}
  \label{1-1-11-toric-ideal-gen-binomial}
  \uses{1-1-10-toric-ideal}
  \lean{AddMonoidAlgebra.isToricIdeal_iff_exists_span_single_sub_single}
  % \leanok

  Proposition 1.1.11: an ideal is toric if and only if it's prime and generated by binomials $x^\alpha - x^\beta$.
\end{proposition}
\begin{proof}
  \uses{1-1-1-subgroup-subtorus, 1-1-9-ideal-ya}
  % \leanok

\end{proof}


\begin{proposition}[The spectrum of an affine monoid algebra is an affine toric variety]
  \label{1-1-14-aff-tor-var-spec-aff-mon-alg}
  \uses{0-aff-mon, 1-1-3-aff-tor-var}
  \lean{AffineToricVarietyFromMonoid.instToricVariety}
  \leanok

  If $S$ is an affine monoid, then $\Spec(\Bbbk[S])$ is an affine toric variety.
\end{proposition}
\begin{proof}
  \uses{0-aff-mon-alg-domain, 0-full-faithful-grp-alg, 1-1-torus-spec}
  % \leanok

  Identify the torus with $\Bbbk[x_1^{\pm1}, \dotsc, x_n^{\pm1}]$ using Lemma \ref{1-1-torus-spec}.
  $i$ induces a morphism $T \to \Spec(\Bbbk[S])$. It's an open embedding as $i$ gives the localization of $\Bbbk[S]$ at $\chi^{a_i}$, so $\im i$ is an affine open. It's dominant as $\Spec(\Bbbk[S])$ is integral and so is irreducible, and $\im i$ is open and nonempty, so dense. The torus action is given by the natural restriction of comultiplication on $\Bbbk[x_1^{\pm1}, \dotsc, x_n^{\pm1}]$ using Proposition \ref{0-full-faithful-grp-alg}.
\end{proof}


\begin{proposition}[The character lattice of the spectrum of an affine monoid algebra]
  \label{1-1-14-char-spec-aff-mon-alg}
  \uses{1-1-14-aff-tor-var-spec-aff-mon-alg, 1-1-char}
  % \lean{}
  % \leanok

  If $S$ is an affine monoid, then the character lattice of $\Spec(\Bbbk[S])$ is $\Z S$.
\end{proposition}
\begin{proof}
  \uses{}
  % \leanok

  It is what it is.
\end{proof}


\begin{proposition}
  \label{1-1-14-spec-aff-mon-alg-eq-ya}
  \uses{0-aff-mon, 1-1-3-aff-tor-var, 1-1-7-ya}
  % \lean{}
  % \leanok

  If $S$ is an affine monoid and $\mathcal A$ is a finite set generating $S$ as a monoid, then $\Spec(\Bbbk[S]) = Y_{\mathcal A}$.
\end{proposition}
\begin{proof}
  \uses{1-1-8-aff-tor-var-ya, 1-1-14-char-spec-aff-mon-alg}
  % \leanok

  We get a $\Bbbk$-algebra homomorphism $\pi : \Bbbk[x_1, \dotsc, x_s] \to \Bbbk[\Z S]$ given by $\mathcal A$; this induces a morphism $\Phi_{\mathcal A} : T \to \Bbbk^s$. The kernel of $\pi$ is the toric ideal of $Y_{\mathcal A}$ and $\pi$ is clearly surjective, so $Y_{\mathcal A} = \mathbb V(\ker(\pi)) = \Spec(\Bbbk[x_1, \dotsc, x_s] / \ker(\pi)) = \Spec(\bbC[S])$.
\end{proof}


\begin{definition}
  \label{torActOnAlg}
  \uses{1-1-torus}
  % \lean{}
  % \leanok

  Torus action on semigroup algebra
\end{definition}


\subsection{Equivalence of Constructions}


\begin{lemma}
  \label{lmm:1-1-16}
  \uses{torActOnAlg}
\end{lemma}
\begin{proof}
  \uses{1-1-2-char-eigenspace-direct-sum}
  % \leanok

\end{proof}


\begin{theorem}
  \label{thm:1-1-17}
  \uses{0-aff-mon, 1-1-3-aff-tor-var, 1-1-7-ya, 1-1-10-lattice-ideal}
  TFAE:
  \begin{enumerate}
    \item $V$ is an affine toric variety.
    \item $V = Y_{\mathcal A}$ for some finite $\mathcal A$.
    \item $V$ is an affine variety defined by a toric ideal.
    \item $V = \Spec \Bbbk[S]$ for an affine monoid $S$.
  \end{enumerate}
\end{theorem}
\begin{proof}
  \uses{torActOnAlg, 1-1-8-aff-tor-var-ya, 1-1-9-ideal-ya, 1-1-14-aff-tor-var-spec-aff-mon-alg, lmm:1-1-16}
  % \leanok

\end{proof}

\section{Cones and Affine Toric Varieties}

\subsection{Convex Polyhedral Cones}

Fix a pair of dual real vector spaces $M$ and $N$.


\begin{definition}[Convex cone generated by a set]
  \label{1-2-1-cone-hull}
  \uses{}
  \lean{Submodule.span}
  \leanok

  For a set $S \subseteq N$, the {\bf cone generated by $S$}, aka {\bf cone hull of $S$}, is
  $$\Cone(S) := \left\{\sum_{u \in S} \lambda_u u | \lambda_u \ge 0\right\}$$
\end{definition}


\begin{definition}[Convex polyhedral cone]
  \label{1-2-1-polyhedral-cone}
  \uses{1-2-1-cone-hull}
  \lean{PointedCone.IsPolyhedral}
  \leanok

  A {\bf polyhedral cone} is a set that can be written as $\Cone(S)$ for some finite set $S$.
\end{definition}


\begin{definition}[Convex hull]
  \label{1-2-2-convex-hull}
  \uses{}
  \lean{convexHull}
  \leanok

  For a set $S \subseteq N$, the {\bf convex hull of $S$} is
  $$\Conv(S) := \left\{\sum_{u \in S} \lambda_u | \lambda_u \ge 0, \sum_u \lambda_u = 1\right\}$$
\end{definition}


\begin{definition}[Polytope]
  \label{1-2-2-polytope}
  \uses{1-2-2-convex-hull}
  \lean{IsPolytope}
  \leanok

  A {\bf polytope} is a set that can be written as $\Conv(S)$ for some finite set $S$.
\end{definition}


\subsection{Dual Cones and Faces}


\begin{definition}[Dual cone]
  \label{1-2-3-dual-cone}
  \uses{1-2-2-convex-hull}
  \lean{PointedCone.dual}
  \leanok

  Given a polyhedral cone $\sigma \subseteq N$, its {\bf dual cone} is defined by
  $$\sigma^\vee = \{m \in M | \forall u \in \sigma, \langle m, u\rangle \ge 0\}$$.
\end{definition}


\begin{proposition}[Dual of a polyhedral cone]
  \label{1-2-4-dual-polyhedral-cone}
  \uses{1-2-1-polyhedral-cone, 1-2-3-dual-cone}
  % \lean{}
  % \leanok

  If $\sigma$ is polyhedral, then its dual $\sigma^\vee$ is polyhedral too.
\end{proposition}
\begin{proof}
  \uses{}
  % \leanok

  Classic. See \cite{Oda_1988} maybe.
\end{proof}


\begin{proposition}[Dual cone of a sumset]
  \label{1-2-dual-cone-add}
  \uses{1-2-3-dual-cone}
  % \lean{}
  % \leanok

  If $\sigma_1, \sigma_2$ are two cones, then
  $$(\sigma_1 + \sigma_2)^\vee = \sigma_1^\vee \cap \sigma_2^\vee.$$
\end{proposition}
\begin{proof}
  \uses{}
  % \leanok

  Classic. See \cite{Oda_1988} maybe.
\end{proof}


\begin{proposition}[Double dual of a polyhedral cone]
  \label{1-2-4-double-dual-polyhedral-cone}
  \uses{1-2-1-polyhedral-cone, 1-2-3-dual-cone}
  % \lean{}
  % \leanok

  If $\sigma$ is polyhedral, then $\sigma^{\vee\vee} = \sigma$.
\end{proposition}
\begin{proof}
  \uses{}
  % \leanok

  Classic. See \cite{Oda_1988} maybe.
\end{proof}


Given $m \ne 0$ in $M$, we get the hyperplane
$$H_m = \{u \in N | \langle m, u\rangle = 0\} \subseteq N$$
and the closed half-space
$$H_m^+ = \{u \in N | \langle m, u\rangle \ge 0\} \subseteq N.$$


\begin{definition}[Face of a cone]
  \label{1-2-5-face}
  \uses{}
  \lean{IsExposed}
  \leanok

  If $\sigma$ is a cone, then a subset of $\sigma$ is a {\bf face} iff it is the intersection of $\sigma$ with some halfspace. We write this $\tau \preceq \sigma$. If furthermore $\tau \ne \sigma$, we call $\tau$ a proper face and write $\tau \prec \sigma$.
\end{definition}


\begin{definition}[Edge of a cone]
  \label{1-2-5-edge}
  \uses{1-2-5-face}
  % \lean{}
  % \leanok

  A dimension 1 face of a cone is called an \emph{edge}.
\end{definition}


\begin{definition}[Facet of a cone]
  \label{1-2-5-facet}
  \uses{1-2-5-face}
  % \lean{}
  % \leanok

  A codimension 1 face of a cone is called a \emph{facet}.
\end{definition}


\begin{lemma}[Face of a polyhedral cone]
  \label{1-2-6-face-polyhedral-cone}
  \uses{1-2-1-polyhedral-cone, 1-2-5-face}
  % \lean{}
  % \leanok

  If $\sigma$ is a polyhedral cone, then every face of $\sigma$ is a polyhedral cone.
\end{lemma}


\begin{lemma}[Intersection of faces]
  \label{1-2-6-inter-faces}
  \uses{1-2-1-polyhedral-cone, 1-2-5-face}
  % \lean{}
  % \leanok

  If $\sigma$ is a polyhedral cone, then the intersection of two faces of $\sigma$ is a face of $\sigma$.
\end{lemma}
\begin{proof}
  \uses{}
  % \leanok

  Classic. See \cite{Oda_1988} maybe.
\end{proof}


\begin{lemma}[Face of a face]
  \label{1-2-6-face-face}
  \uses{1-2-1-polyhedral-cone, 1-2-5-face}
  % \lean{}
  % \leanok

  A face of a face of a polyhedral cone $\sigma$ is again a face of $\sigma$.
\end{lemma}
\begin{proof}
  \uses{}
  % \leanok

  Classic. See \cite{Oda_1988} maybe.
\end{proof}


\begin{lemma}
  \label{1-2-7-face-mem-of-add}
  \uses{1-2-1-polyhedral-cone, 1-2-5-face}
  % \lean{}
  % \leanok

  Let $\tau$ be a face of a polyhedral cone $\sigma$. If $v, w \in \sigma$ and $v + w \in \tau$, then $v, w \in \tau$.
\end{lemma}
\begin{proof}
  \uses{}
  % \leanok

  Classic. See \cite{Oda_1988} maybe.
\end{proof}


\begin{proposition}[Dual cone of the intersection of halfspaces]
  \label{1-2-8-dual-cone-inter-halfspaces}
  \uses{1-2-1-cone-hull, 1-2-3-dual-cone}
  % \lean{}
  % \leanok

  If $\sigma = H_{m_1}^+ \cap \dots \cap H_{m_s}^+$, then
  $$\sigma^\vee = \Cone(m_1, \dots, m_s).$$
\end{proposition}
\begin{proof}
  \uses{}
  % \leanok

  Classic. See \cite{Oda_1988} maybe.
\end{proof}


\begin{proposition}[Facets of a full dimensional cone]
  \label{1-2-8-facet-full-dim-cone}
  \uses{1-2-1-cone-hull, 1-2-5-facet}
  % \lean{}
  % \leanok

  If $\sigma$ is a full dimensional cone, then facets of $\sigma$ are of the form $H_m \cap \sigma$.
\end{proposition}
\begin{proof}
  \uses{}
  % \leanok

  Classic. See \cite{Oda_1988} maybe.
\end{proof}


\begin{proposition}[Intersection of facets containing a face]
  \label{1-2-8-inter-facet}
  \uses{1-2-5-facet}
  % \lean{}
  % \leanok

  Every proper face $\tau \prec \sigma$ of a polyhedral cone $\sigma$ is the intersection of the facets of $\sigma$ containing $\tau$.
\end{proposition}
\begin{proof}
  \uses{}
  % \leanok

  Classic. See \cite{Oda_1988} maybe.
\end{proof}


\begin{definition}[Dual face]
  \label{1-2-10-dual-face}
  \uses{1-2-3-dual-cone, 1-2-5-face}
  % \lean{}
  % \leanok

  Given a cone $\sigma$ and a face $\tau \preceq \sigma$, the {\bf dual face} to $\tau$ is
  $$\tau^* := \sigma^\vee \cap \tau^\perp$$
\end{definition}


\begin{proposition}[The dual face is a face of the dual]
  \label{1-2-10-dual-face-face-dual}
  \uses{1-2-10-dual-face}
  % \lean{}
  % \leanok

  If $\tau \preceq \sigma$, then $\tau^* \preceq \sigma^\vee$.
\end{proposition}
\begin{proof}
  \uses{}
  % \leanok

  Classic. See \cite{Oda_1988} maybe.
\end{proof}


\begin{proposition}[The double dual of a face]
  \label{1-2-10-double-dual-face-dual-face}
  \uses{1-2-10-dual-face}
  % \lean{}
  % \leanok

  If $\tau \preceq \sigma$, then $\tau^{**} = \tau$.
\end{proposition}
\begin{proof}
  \uses{1-2-4-double-dual-polyhedral-cone}
  % \leanok

  Classic. See \cite{Oda_1988} maybe.
\end{proof}


\begin{proposition}[The dual of a face is antitone]
  \label{1-2-10-dual-face-antitone}
  \uses{1-2-10-dual-face}
  % \lean{}
  % \leanok

  If $\tau' \preceq \tau \preceq \sigma$, then $\tau' \preceq \tau$.
\end{proposition}
\begin{proof}
  \uses{}
  % \leanok

  Classic. See \cite{Oda_1988} maybe.
\end{proof}


\begin{proposition}[The dimension of the dual of a face]
  \label{1-2-10-double-dual-face-dual-face}
  \uses{1-2-10-dual-face}
  % \lean{}
  % \leanok

  If $\tau \preceq \sigma$, then
  $$\dim \tau + \dim \tau^* = \dim N.$$
\end{proposition}
\begin{proof}
  \uses{}
  % \leanok

  Classic. See \cite{Oda_1988} maybe.
\end{proof}


\subsection{Relative Interiors}


\begin{definition}[Relative interior]
  \label{1-2-rel-interior}
  \uses{}
  \lean{intrinsicInterior}
  \leanok

  The {\bf relative interior}, aka {\bf intrinsic interior}, of a cone $\sigma$ is the interior of $\sigma$ as a subset of its span.
\end{definition}


\begin{lemma}[The relative interior in terms of the inner product]
  \label{1-2-rel-interior-inner}
  \uses{1-2-rel-interior}
  % \lean{}
  % \leanok

  For a cone $\sigma$,
  $$u \in \Relint(\sigma) \iff \forall m \in \sigma^\vee \setminus \sigma^\perp, \langle m, u\rangle > 0$$
\end{lemma}
\begin{proof}
  \uses{}
  % \leanok

  Classic. See \cite{Oda_1988} maybe.
\end{proof}


\begin{lemma}[Relative interior of a dual face]
  \label{1-2-rel-interior-dual-face}
  \uses{1-2-10-dual-face, 1-2-rel-interior}
  % \lean{}
  % \leanok

  If $\tau \preceq \sigma$ and $m \in \sigma^\vee$, then
  $$m \in \Relint(\tau^*) \iff \tau = H_m \cap \sigma$$
\end{lemma}
\begin{proof}
  \uses{}
  % \leanok

  Classic. See \cite{Oda_1988} maybe.
\end{proof}


\begin{lemma}[Minimal face of a cone]
  \label{1-2-min-face}
  \uses{1-2-5-face, 1-2-rel-interior}
  % \lean{}
  % \leanok

  If $\sigma$ is a cone, then $W := \sigma \cap (-\sigma)$ is a subspace. Furthermore,
  $W = H_m \cap \sigma$ whenever $m \in \Relint(\sigma^\vee)$.
\end{lemma}
\begin{proof}
  \uses{}
  % \leanok

  Classic. See \cite{Oda_1988} maybe.
\end{proof}


\subsection{Strong Convexity}


\begin{definition}[Salient cones]
  \label{1-2-12-salient-cone}
  \uses{}
  \lean{ConvexCone.Salient}
  \leanok

  A cone $\sigma$ is {\bf salient}, aka {\bf pointed} or {\bf strongly convex}, if $\sigma \cap (-\sigma) = \{0\}$.
\end{definition}


\begin{proposition}[Alternative definitions of salient cones]
  \label{1-2-12-salient-cone-tfae}
  \uses{1-2-3-dual-cone, 1-2-12-salient-cone}
  % \lean{}
  % \leanok

  The following are equivalent
  \begin{enumerate}
    \item $\sigma$ is salient
    \item $\{0\} \preceq \sigma$
    \item $\sigma$ contains no positive dimensional subspace
    \item $\dim \sigma^\vee = \dim N$
  \end{enumerate}
\end{proposition}
\begin{proof}
  \uses{}
  % \leanok

  Classic. See \cite{Oda_1988} maybe.
\end{proof}


\subsection{Separation}


\begin{lemma}[Separation lemma]
  \label{1-2-13-separation-lemma}
  \uses{1-2-1-polyhedral-cone, 1-2-5-face}
  % \lean{}
  % \leanok

  Let $\sigma_1, \sigma_2$ be polyhedral cones meeting along a common face $\tau$. Then
  $$\tau = H_m \cap \sigma_1 = H_m \cap \sigma_2$$
  for any $m \in \Relint(\sigma_1^\vee \cap (-\sigma_2)^\vee)$.
\end{lemma}
\begin{proof}
  \uses{1-2-dual-cone-add, 1-2-min-face}
  % \leanok

  See \cite{Cox_2011}.
\end{proof}


\subsection{Rational Polyhedral Cones}


Let $M$ and $N$ be dual lattices with associated vector spaces $M_\R := M \ox_\Z \R, N_\R := N \ox_\Z \R$.


\begin{definition}[Rational cone]
  \label{1-2-14-rat-cone}
  \uses{1-2-1-cone-hull}
  % \lean{}
  % \leanok

  A cone $\sigma \subseteq N_\R$ is {\bf rational} if $\sigma = \Cone(S)$ for some finite set $S \subseteq N$.
\end{definition}


\begin{lemma}[Faces of a rational cone]
  \label{1-2-14-face-rat-cone}
  \uses{1-2-5-face, 1-2-14-rat-cone}
  % \lean{}
  % \leanok

  If $\tau \preceq \sigma$ is a face of a rational cone, then $\tau$ itself is rational.
\end{lemma}
\begin{proof}
  \uses{}
  % \leanok

  Classic. See \cite{Oda_1988} maybe.
\end{proof}


\begin{lemma}[The dual of a rational cone]
  \label{1-2-14-dual-rat-cone}
  \uses{1-2-3-dual-cone, 1-2-14-rat-cone}
  % \lean{}
  % \leanok

  $\sigma^\vee$ is a rational cone iff $\sigma$ is.
\end{lemma}
\begin{proof}
  \uses{}
  % \leanok

  Classic. See \cite{Oda_1988} maybe.
\end{proof}


\begin{definition}[Ray generator]
  \label{1-2-ray-gen}
  \uses{1-2-5-edge, 1-2-14-rat-cone}
  % \lean{}
  % \leanok

  If $\rho$ is an edge of a rational cone $\sigma$, then the monoid $\rho \cap N$ is generated by a unique element $u_\rho \in \rho \cap N$, which we call the {\bf ray generator} of $\rho$.
\end{definition}


\begin{definition}[Minimal generators]
  \label{1-2-min-gen}
  \uses{1-2-ray-gen}
  % \lean{}
  % \leanok

  The {\bf minimal generators} of a rational cone $\sigma$ are the ray generators of its edges.
\end{definition}


\begin{lemma}[A rational cone is generated by its minimal generators]
  \label{1-2-15-cone-hull-min-gen}
  \uses{1-2-12-salient-cone, 1-2-min-gen}
  % \lean{}
  % \leanok

  A salient convex rational polyhedral cone is generated by its minimal generators.
\end{lemma}
\begin{proof}
  \uses{}
  % \leanok

  Classic. See \cite{Oda_1988} maybe.
\end{proof}


\begin{definition}[Regular cone]
  \label{1-2-16-reg-cone}
  \uses{1-2-min-gen}
  % \lean{}
  % \leanok

  A salient rational polyhedral cone $\sigma$ is {\bf regular}, aka {\bf smooth}, if its minimal generators form part of a $\Z$-basis of $N$.
\end{definition}


\begin{definition}[Simplicial cone]
  \label{1-2-16-simplicial-cone}
  \uses{1-2-min-gen}
  % \lean{}
  % \leanok

  A salient rational polyhedral cone $\sigma$ is {\bf simplicial} if its minimal generators are $\R$-linearly independent.
\end{definition}


\subsection{Semigroup Algebras and Affine Toric Varieties}


\begin{definition}[Dual lattice of a cone]
  \label{1-2-17-dual-lat-cone}
  \uses{1-2-3-dual-cone}
  % \lean{}
  % \leanok

  If $\sigma \subseteq N_\R$ is a polyhedral cone, then the lattice points
  \[
    S_\sigma := \sigma^\vee \cap M
  \]
  form a monoid.
\end{definition}


\begin{proposition}[Gordan's lemma]
  \label{1-2-17-gordan-lemma}
  \uses{1-2-17-dual-lat-cone}
  % \lean{}
  % \leanok

  $S_\sigma$ is finitely generated as a monoid.
\end{proposition}
\begin{proof}
  \uses{1-2-14-dual-rat-cone}
  % \leanok

  See \cite{Cox_2011}.
\end{proof}


\begin{definition}[Affine toric variety of a rational polyhedral cone]
  \label{1-2-18-aff-tor-var-rat-polyhedral-cone}
  \uses{1-1-3-aff-tor-var, 1-2-17-dual-lat-cone}
  % \lean{}
  % \leanok

  $U_\sigma := \Spec \bbC[S_\sigma]$ is an affine toric variety.
\end{definition}


\begin{theorem}[Dimension of the affine toric variety of a rational polyhedral cone]
  \label{1-2-18-dim-aff-tor-var-rat-polyhedral-cone}
  \uses{1-2-12-salient-cone, 1-2-18-aff-tor-var-rat-polyhedral-cone}
  % \lean{}
  % \leanok

  \[
    \dim U_\sigma = \dim N \iff \text{ the torus of $U_\sigma$ is } T_N = N \ox_[\Z] \bbC^* \iff \sigma \text{ is salient}.
  \]
\end{theorem}
\begin{proof}
  \uses{1-1-14-char-spec-aff-mon-alg, 1-2-12-salient-cone-tfae, 1-2-17-gordan-lemma}
  % \leanok

  See \cite{Cox_2011}.
\end{proof}


\begin{proposition}[The irreducible elements of the dual lattice of a cone]
  \label{1-2-22-irred-dual-lat}
  \uses{0-irred, 1-2-min-gen, 1-2-12-salient-cone, 1-2-17-dual-lat-cone}
  % \lean{}
  % \leanok

  If $\sigma \subseteq N_\R$ is salient of maximal dimension, then the irreducible elements of $S_\sigma$ are precisely the minimal generators of $\sigma^\vee$.
\end{proposition}
\begin{proof}
  \uses{0-irred-subset-gen, 0-irred-gen}
  % \leanok

  See \cite{Cox_2011}.
\end{proof}

% \input{chapters/01-3-properties-of-affine-toric-varieties.tex}

% \chapter{Projective Toric Varieties}

% \input{chapters/02-1-lattice-points-and-projective-toric-varieties.tex}
% \input{chapters/02-2-lattice-points-and-polytopes.tex}
% \input{chapters/02-3-polytopes-and-projective-toric-varieties.tex}
% \input{chapters/02-4-properties-of-projective-toric-varieties.tex}

% \chapter{Normal Toric Varieties}

% \input{chapters/03-1-fans-and-normal-toric-varieties.tex}
% \input{chapters/03-2-orbit-cone-correspondence.tex}
% \input{chapters/03-3-toric-morphisms.tex}
% \input{chapters/03-4-complete-and-proper.tex}

\bibliographystyle{plain} % We choose the "plain" reference style
\bibliography{Toric}

