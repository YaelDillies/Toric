% In this file you should put the actual content of the blueprint.
% It will be used both by the web and the print version.
% It should *not* include the \begin{document}
%
% If you want to split the blueprint content into several files then
% the current file can be a simple sequence of \input. Otherwise It
% can start with a \section or \chapter for instance.

\setcounter{chapter}{-1} % Make Prerequisites be Chapter 0

\chapter{Prerequisites}

\section{Affine Monoid}


\begin{definition}[Affine monoid]
  \label{0-affine-monoid}
  \lean{AddCancelCommMonoid, AddMonoid.FG, AddMonoid.IsTorsionFree}
  \leanok

  An \emph{affine monoid} is a finitely generated commutative monoid which is:
  \begin{itemize}
    \item cancellative: if $a + c = b + c$ then $a = b$, and
    \item torsion-free: if $n a = n b$ then $a = b$ (for $n \geq 1$).
  \end{itemize}
\end{definition}


\begin{proposition}[Embedding an affine monoid inside a lattice]
  \label{0-embed-affine-monoid}
  \uses{0-affine-monoid}
  % \lean{}
  % \leanok

  If $S$ is an affine monoid, then $S$ can be embedded inside $\Z^n$ for some $n$.
\end{proposition}
\begin{proof}
  \uses{}
  % \leanok

  Embed $S$ inside its Grothendieck group $G$. Prove that $G$ is finitely generated free.
\end{proof}


\begin{definition}[Irreducible element]
  \label{0-irred}
  \uses{}
  % \lean{}
  % \leanok

  An element $x$ of a monoid $S$ is \emph{irreducible} if $x = y + z$ implies $y = 0$ or $z = 0$.
\end{definition}


\begin{definition}[Hilbert basis]
  \label{0-hilbert-basis}
  \uses{0-irred}
  % \lean{}
  % \leanok

  The \emph{Hilbert basis} of the monoid $S$ is
  \[
    \mathcal H_S := \{x \in S | x \text{ is irreducible}\}
  \]
\end{definition}


\begin{proposition}[Finiteness of the Hilbert basis]
  \label{0-hilbert-basis-finite}
  \uses{0-hilbert-basis}
  % \lean{}
  % \leanok

  If no non-zero element of the finitely generated monoid $S$ is a unit, then $\mathcal H_S$ generates $S$, is finite, and is the minimal generating set of $S$.
\end{proposition}
\begin{proof}
  \uses{}
  % \leanok

  See \cite{Cox_2011}.
\end{proof}

\section{Tensor Product}


\begin{lemma}[The tensor product of linearly independent families]
  \label{0-tensor-lin-indep}
  \uses{}
  \lean{LinearIndependent.tmul}
  \leanok

  If $f$ and $g$ are linearly independent families of points in semimodules $M$ and $N$, then $i j \mapsto f i \ox g j$ is a linearly independent family of points in $M \ox N$.
\end{lemma}
\begin{proof}
  \uses{}
  % \leanok

  Assume
  \[
    \sum_{i, j} c_{i, j} f i \ox g j = \sum_{i, j} d_{i, j} f i \ox g j
  \]
  Then
  \[
    \sum_i f i \ox \left(\sum_j c_{i, j} g j\right) = \sum_i f i \ox \left(\sum_j d_{i, j} g j\right)
  \]
  Since $f$ is linearly independent,
  \[
    \sum_j c_{i, j} g j = \sum_j d_{i, j} g j
  \]
  for every $i$. Since $g$ is linearly independent, $c_{i, j} = d_{i, j}$ for all $i, j$, as wanted.
\end{proof}

\section{Hopf algebras}


\subsection{Group-like elements}


\begin{definition}[Group-like elements]
  \label{0-grp-like}
  \uses{}
  \lean{Coalgebra.IsGroupLikeElem}
  \leanok

  An element $a$ of a coalgebra $A$ is \emph{group-like} if it is a unit and $\Delta(a) = a \ox a$, where $\Delta$ is the comultiplication map.
\end{definition}


\begin{lemma}[Bialgebra homs preserve group-like elements]
  \label{0-grp-like-map}
  \uses{0-grp-like}
  \lean{Coalgebra.IsGroupLikeElem.map}
  \leanok

  Let $f : A \to B$ be a bi-algebra hom. If $a \in A$ is group-like, then $f(a)$ is group-like too.
\end{lemma}
\begin{proof}
  \uses{}
  \leanok

  $a$ is a unit, so $f(a)$ is a unit too. Then
  \[
    f(a) \ox f(a) = (f \ox f)(\Delta_A(a)) = \Delta_B(f(a))
  \]
  so $f(a)$ is group-like.
\end{proof}


\begin{lemma}[Independence of group-like elements]
  \label{0-grp-like-lin-indep}
  \uses{0-grp-like}
  \lean{Coalgebra.linearIndepOn_isGroupLikeElem}
  \leanok

  The group-like elements in $A$ are linearly independent.
\end{lemma}
\begin{proof}
  \uses{0-tensor-lin-indep}
  \leanok

  See Lemma 4.23 in \cite{Milne_2017}.
\end{proof}


\begin{lemma}[Group-like elements in a group algebra]
  \label{0-grp-like-grp-alg}
  \uses{0-grp-like}
  \lean{MonoidAlgebra.isGroupLikeElem_iff_mem_range_of}
  \leanok

  The group-like elements of $k[M]$ are exactly the image of $M$.
\end{lemma}
\begin{proof}
  \uses{0-grp-like-lin-indep}
  \leanok

  See Lemma 12.4 in \cite{Milne_2017}.
\end{proof}


\subsection{The group algebra functor}


\begin{proposition}[The antipode is a antihomomorphism]
  \label{0-hopf-antipode-antihom}
  \uses{}
  \lean{HopfAlgebra.antipode_mul_antidistrib, HopfAlgebra.antipode_mul_distrib}
  \leanok

  If $A$ is a $R$-Hopf algebra, then the antipode map $s : A \to A$ is anti-commutative, ie $s(a * b) = s(b) * s(a)$. If further $A$ is commutative, then $s(a * b) = s(a) * s(b)$.
\end{proposition}
\begin{proof}
  \uses{}
  % \leanok

  Any standard reference will have a proof.
\end{proof}


\begin{proposition}[Hopf algebras are cogroup objects in the category of algebras]
  \label{0-hopf-cogrp-alg}
  \uses{}
  \lean{grp_Class_op_commAlgOf, isMon_Hom_commAlgOfHom, bialgebra_unop, hopfAlgebra_unop, IsMon_Hom.toBialgHom}
  \leanok

  From a $R$-Hopf algebra, one can build a cogroup object in the category of $R$-algebras.

  From a cogroup object in the category of $R$-algebras, one can build a $R$-Hopf algebra.
\end{proposition}
\begin{proof}
  \uses{0-hopf-antipode-antihom}
  \leanok

  Turn the arrows around.
\end{proof}


\begin{definition}[The group algebra functor]
  \label{0-grp-alg}
  \uses{0-hopf-cogrp-alg}
  \lean{commGrpAlgebra}
  \leanok

  For a commutative ring $R$, we have a functor $G \rightsquigarrow R[G] : \Grp \to \Hopf_R$.
\end{definition}


\begin{proposition}[The group algebra functor is fully faithful]
  \label{0-full-faithful-grp-alg}
  \uses{0-grp-alg}
  \lean{MonoidAlgebra.mapDomainBialgHomMulEquiv, AddMonoidAlgebra.mapDomainBialgHomMulEquiv, commGrpAlgebra.fullyFaithful, commGrpAlgebra.instFull, commGrpAlgebra.instFaithful}
  \leanok

  For a field $K$, the functor $G \rightsquigarrow K[G]$ from the category of groups to the category of Hopf algebras over $K$ is fully faithful.
\end{proposition}
\begin{proof}
  \uses{0-grp-like-grp-alg, 0-grp-like-map}
  \leanok

  It is clearly faithful.
  Now for the full part, if $f : K[G] \to K[H]$ is a Hopf algebra hom, then we get a series of maps
  \[
    G \simeq \text{ group-like elements of } R[G] \to \text{ group-like elements of } R[H] \simeq H
  \]
  and each map separately is clearly multiplicative.
\end{proof}

\section{Group Schemes}


\subsection{Correspondence between Hopf algebras and affine group schemes}


We want to show that Hopf algebras correspond to affine group schemes.
This can easily be done categorically assuming both categories on either side are defined thoughtfully.
However, the categorical version will not be workable with if we do not also have links to the non-categorical notions.
Therefore, one solution would be to build the left, top and right edges of the following
diagram so that the bottom edge can be obtained by composing the three:
%   Cogrp Mod_R   ⥤        Grp Sch_{Spec R}
%       ↑ ↓                        ↓
% R-Hopf algebras → Affine group schemes over Spec R


\subsubsection{Bundling/unbundling Hopf algebras}


We have already done the left edge in the previous section.


\subsubsection{Spec of a Hopf algebra}


Now let's do the top edge.


\begin{proposition}[Sliced adjoint functors]
  \label{0-slice-adj}
  \uses{}
  % \lean{}
  % \leanok

  If $a : F \vdash G$ is an adjunction between $F : C \to D$ and $G : D \to C$ and $X : C$, then there is an adjunction between $F / X : C / X \to D / F(X)$ and $G / X : D / F(X) \to C / X$.
\end{proposition}
\begin{proof}
  \uses{}
  % \leanok

  See https://ncatlab.org/nlab/show/sliced+adjoint+functors+--+section.
\end{proof}


\begin{proposition}[Limit-preserving functors lift to over categories]
  \label{0-over-lim}
  \uses{}
  \lean{CategoryTheory.Limits.PreservesLimitsOfShape.overPost, CategoryTheory.Limits.PreservesLimitsOfSize.overPost, CategoryTheory.Limits.PreservesFiniteProducts.overPost}
  \leanok

  If $F : C \to D$ is a functor preserving limits of shape $J$, then so is the obvious functor $C / X \to D / F(X)$.
\end{proposition}
\begin{proof}
  \uses{}
  % \leanok

  Hopefully easy.
\end{proof}


\begin{proposition}[Fully faithful product-preserving functors lift to group objects]
  \label{0-full-faithful-grp}
  \uses{}
  \lean{CategoryTheory.Functor.Faithful.mapGrp, CategoryTheory.Functor.Full.mapGrp}
  \leanok

  If a finite-products-preserving functor $F : C \to D$ is fully faithful, then so is $\Grp(F) : \Grp C \to \Grp D$.
\end{proposition}
\begin{proof}
  \uses{}
  \leanok

  Faithfulness is immediate.

  For fullness, assume $f : F(G) \to F(H)$ is a morphism. By fullness of $F$, find $g : G \to H$ such that $F(g) = f$. $g$ is a morphism because we can pull back each diagram from $D$ to $C$ along $F$ which is faithful.
\end{proof}


\begin{definition}[Spec as a functor on algebras]
  \label{0-spec-alg}
  \uses{}
  \lean{algSpec}
  \leanok

  Spec is a contravariant functor from the category of $R$-algebras to the category of schemes over $\Spec_R$.
\end{definition}


\begin{proposition}[Spec as a functor on algebras is fully faithful]
  \label{0-full-faithful-spec-alg}
  \uses{0-spec-alg}
  \lean{algSpec.instPreservesLimits, algSpec.instFull, algSpec.instFaithful, algSpec.fullyFaithful}
  \leanok

  Spec is a fully faithful contravariant functor from the category of $R$-algebras to the category of schemes over $\Spec_R$, preserving all limits.
\end{proposition}
\begin{proof}
  \uses{0-slice-adj, 0-over-lim}
  \leanok

  $\Spec : \Ring \to \Sch$ is a fully faithful contravariant functor which preserves all limits, hence so is $\Spec : \Ring_R \to \AffSch_{\Spec R}$ by Proposition \ref{0-over-lim} (alternatively, by Proposition \ref{0-slice-adj}).
\end{proof}


\begin{definition}[Spec as a functor on Hopf algebras]
  \label{0-spec-hopf}
  \uses{0-full-faithful-spec-alg}
  \lean{hopfSpec}
  \leanok

  Spec is a fully faithful contravariant functor from the category of $R$-algebras to the category of group schemes over $\Spec_R$.
\end{definition}


\begin{proposition}[Spec as a functor on Hopf algebras is fully faithful]
  \label{0-full-faithful-spec-hopf}
  \uses{0-spec-alg}
  \lean{hopfSpec.instFull, hopfSpec.instFaithful, hopfSpec.fullyFaithful}
  \leanok

  Spec is a fully faithful contravariant functor from the category of $R$-Hopf algebras to the category of group schemes over $\Spec_R$.
\end{proposition}
\begin{proof}
  \uses{0-full-faithful-grp, 0-full-faithful-spec-alg}
  \leanok

  $\Spec : \Ring_R \to \Sch_{\Spec R}$ is a fully faithful contravariant functor preserving all limits according to Proposition \ref{0-spec-alg}, therefore $\Spec : \Hopf_R \to \GrpSch_{\Spec R}$ too is fully faithful according to \ref{0-full-faithful-grp}.
\end{proof}


\subsection{Essential image of Spec on Hopf algebras}


Finally, let's do the right edge.


\begin{proposition}[Essential image of a sliced functor]
  \label{0-ess-image-over}
  \uses{}
  % \lean{}
  % \leanok

  If $F : C \to D$ is a fully faithful functor between cartesian-monoidal categories, then $F / X : C / X \hom D / F(X)$ has the same essential image as $F$.
\end{proposition}
\begin{proof}
  \uses{}
  % \leanok

  Transfer all diagrams.
\end{proof}


\begin{proposition}[Equivalences lift to group object categories]
  \label{0-grp-equiv}
  \uses{}
  \lean{CategoryTheory.Equivalence.mapGrp}
  \leanok

  If $e : C \backsimeq D$ is an equivalence of cartesian-monoidal categories, then $\Grp(e) : \Grp(C) \backsimeq \Grp(D)$ too is an equivalence of categories.
\end{proposition}
\begin{proof}
  \uses{}
  \leanok

  Transfer all diagrams.
\end{proof}


\begin{proposition}[Essential image of a functor on group objects]
  \label{0-ess-image-grp}
  \uses{}
  \lean{CategoryTheory.Functor.essImage_mapGrp}
  \leanok

  If $F : C \to D$ is a fully faithful functor between cartesian-monoidal categories, then $\Grp(F) : \Grp(C) \hom \Grp(D)$ has the same essential image as $F$.
\end{proposition}
\begin{proof}
  \uses{0-grp-equiv}
  % \leanok

  Transfer all diagrams.
\end{proof}


\begin{proposition}[Essential image of Spec on algebras]
  \label{0-ess-image-spec-alg}
  \uses{0-spec-alg}
  \lean{essImage_algSpec}
  \leanok

  The essential image of $\Spec : \Ring_R \to \Sch_{\Spec R}$ is precisely affine schemes over $\Spec R$.
\end{proposition}
\begin{proof}
  \uses{0-ess-image-over}
  % \leanok

  Direct consequence of Proposition \ref{0-ess-image-over}.
\end{proof}


\begin{proposition}[Essential image of Spec on Hopf algebras]
  \label{0-ess-image-spec-hopf}
  \uses{0-spec-hopf}
  \lean{essImage_hopfSpec}
  \leanok

  The essential image of $\Spec : \Hopf_R \to \GrpSch_{\Spec R}$ is precisely affine group schemes over $\Spec R$.
\end{proposition}
\begin{proof}
  \uses{0-ess-image-grp, 0-ess-image-spec-alg, 0-full-faithful-spec-alg}
  \leanok

  Direct consequence of Propositions \ref{0-ess-image-grp} and \ref{0-ess-image-spec-alg}.
\end{proof}


\subsection{Diagonalisable groups}


\begin{definition}
  \label{0-spec-grp-alg}
  \uses{0-spec-hopf}
  \lean{hopfSpec, MonoidAlgebra}
  \leanok

  For a commutative group $G$ we define $D_R(G)$ as the spectrum $\Spec R[G]$ of the group algebra $R[G]$.
\end{definition}


\begin{definition}
  \label{0-diag}
  \uses{0-spec-grp-alg}
  \lean{AlgebraicGeometry.Scheme.IsDiagonalisable}
  % \leanok

  An algebraic group $G$ over $\Spec R$ is {\bf diagonalisable} if it is isomorphic to $D_R(G)$ for some commutative group $G$.
\end{definition}


\begin{theorem}
  \label{0-diag-iff-grp-like-span}
  \uses{0-spec-grp-alg, 0-diag}
  \lean{AlgebraicGeometry.Scheme.isDiagonalisable_iff_span_isGroupLikeElem_eq_top}
  \leanok

  An algebraic group $G$ over a field $k$ is diagonalizable if and only if group-like elements span $\Gamma(G)$.
\end{theorem}
\begin{proof}
  \uses{0-grp-like-lin-indep}
  % \leanok

  See Theorem 12.8 in \cite{Milne_2017}.
\end{proof}


\begin{theorem}
  \label{0-full-faithful-spec-grp-alg}
  \uses{0-spec-grp-alg}
  \lean{specCommGrpAlgebra.fullyFaithful, specCommGrpAlgebra.instFull, specCommGrpAlgebra.instFaithful}
  \leanok

  For a field $k$, $D_k$ is a fully faithful contravariant functor from the category of commutative groups to the category of group schemes over $\Spec k$.
\end{theorem}
\begin{proof}
  \uses{0-full-faithful-spec-hopf, 0-full-faithful-grp-alg}
  \leanok

  Compose Propositions \ref{0-full-faithful-spec-hopf} and \ref{0-full-faithful-grp-alg}.

  Also see Theorem 12.9(a) in \cite{Milne_2017}. See SGA III Exposé VIII for a proof that works for $R$ an arbitrary commutative ring in place of $k$.
\end{proof}


\chapter{Affine Toric Varieties}

\section{Introduction to Affine Toric Varieties}


\subsection{The Torus}


\begin{definition}[The split torus]
  \label{1-1-torus}
  \lean{AlgebraicGeometry.Scheme.SplitTorus, AlgebraicGeometry.Scheme.SplitTorus.instCanonicallyOver}
  \leanok

  The split torus $\Gm^n$ over a scheme $S$ is the pullback of
  $\Spec \Z[x_1^{\pm 1}, \dotsc, x_n^{\pm 1}]$ along the unique map $S \to \Spec \Z$.
\end{definition}


\begin{lemma}[The split torus over $\Spec R$]
  \label{1-1-torus-spec}
  \uses{1-1-torus}
  \lean{AlgebraicGeometry.Scheme.splitTorusIsoSpec, AlgebraicGeometry.Scheme.splitTorusIsoSpecOver}
  \leanok

  The split torus over $\Spec R$ is isomorphic to $\Spec(R[x_1^{\pm 1}, \dotsc, x_n^{\pm 1}])$.
\end{lemma}
\begin{proof}
  \uses{}
  % \leanok

  Ask any toddler on the street.
\end{proof}


\begin{definition}[Characters of a group scheme]
  \label{1-1-char}
  \uses{1-1-torus}
  \lean{AlgebraicGeometry.Scheme.Over.char}
  \leanok

  For a group scheme $G$ over $S$, the {\bf character lattice} of $G$ is
  \[
    X(G) := \Hom_{\mathsf{GrpSch}_S}(G, \Gm).
  \]
  An element $X(G)$ is (unsurprisingly) called a {\bf character}.
\end{definition}

\begin{proposition}[Character lattice of the torus]
  \label{1-1-char-torus}
  \uses{1-1-char}
  % \lean{}
  % \leanok

  Characters of the torus over a field $k$ are isomorphic to $\Z^n$. $X(\Gm^n) = \Z^n$.
\end{proposition}
\begin{proof}
  \uses{0-full-faithful-spec-grp-alg, 1-1-torus-spec}
  % \leanok

  By Propositions \ref{1-1-torus-spec} and \ref{0-full-faithful-spec-grp-alg} in turn, we have
  \[
    X(\Gm^n) = \Hom_{\mathsf{GrpSch}}(\Gm^n, \Gm) = \Hom(k[\Z], k[\Z^n]) = \Hom(\Z, \Z^n) = \Z^n.
  \]
\end{proof}


\begin{proposition}[The image of a torus is a torus]
  \label{1-1-1-group-hom-subtorus} % Proposition 1.1.1(a)
  \uses{1-1-torus}
  % \lean{}
  % \leanok

  Let $T_1$ and $T_2$ be split tori over a field $k$ and let $\Phi: T_1 \to T_2$ be a homomorphism,
  then $\Phi$ factors as
  \[
    T_1 \xrightarrow{\Phi} T_2 = T_1 \xrightarrow{\phi} T \xrightarrow{\iota} T_2,
  \]
  where $T$ is a split torus, $\iota$ is a closed subgroup embedding and $\phi$ is an fpqc homomorphism.
\end{proposition}
\begin{proof}
  \uses{0-full-faithful-spec-grp-alg, 1-1-torus-spec}
  % \leanok
  Let $M_1=X(T_1), M_2=X(T_2)$. Define $M$ to be the image of the homomorphism $M_2 \to M_1$
  corresponding to $\Phi$ and take $T = D_k(M)$. The homomorphisms $\iota,\phi$ correspond to the
  canonical quotient map $M_2 \to M$ and the canonical inclusion $M \to M_1$ respectively.
  Hence $\Phi = \iota\circ\phi$.

  $M$ is a subgroup of a finitely-generated free abelian group $M_1$,
  hence itself a finitely-generated free abelian group. Thus $T$ is a split torus.

  $\iota$ is a closed embedding since the corresponding ring map is a quotient map with kernel
  generated by the kernel of $M_2 \to M_1$.

  $\phi$ is affine, hence quasi-compact. A collection of coset representatives for $M /M_1$
  gives a basis for $k[M]$ as a $k[M_1]$ module, hence $\phi$ is faithfully flat.
\end{proof}


\begin{proposition}[A subgroup of a torus is a torus]
  \label{1-1-1-subgroup-subtorus}
  \uses{1-1-torus}
  % \lean{}
  % \leanok

  Let $T$ be a split torus.
  If $H \subseteq T$ is an irreducible subgroup, then $H$ is a split torus.
\end{proposition}
\begin{proof}
  \uses{0-full-faithful-spec-grp-alg, 1-1-torus-spec}
  % \leanok

  Let $M = X(T), N=X(H)$. Since $H$ is a closed subscheme $M \to N$ is surjective,
  so $N$ is a finitely-generated abelian group. Since $H$ is irreducible it is connected,
  so $N$ is torsion-free, hence free. Thus $H$ is a split torus.
\end{proof}


\begin{definition}[The character eigenspace]
  \label{1-1-char-eigenspace}
  \uses{1-1-char}
  % \lean{}
  % \leanok

  For a finite dimensional representation of a torus $T$ on $W$, the {\bf character eigenspace} of a character $\chi \in X(T)$ is
  \[
    W_m = \{w\in W : t\cdot w = \chi(t)\text{ for all } t\in T \}.
  \]
\end{definition}


\begin{proposition}[Decomposition into character eigenspaces]
  \label{1-1-2-char-eigenspace-direct-sum}
  \uses{1-1-char-eigenspace}

  The space decomposes into the direct sum of the character eigenspaces.
\end{proposition}
\begin{proof}
  \uses{}
  % \leanok

  TODO
\end{proof}

\begin{definition}
  \label{1-1-cochar}
  \uses{1-1-torus}
  \lean{AlgebraicGeometry.Scheme.Over.cochar}
  \leanok

  For a group scheme $G$, the {\bf cocharacter lattice} of $G$ is $\Hom_{\mathsf{GrpSch}_S}(\Gm, G)$.
  An element is called a {\bf cocharacter} or {\bf one-parameter subgroup}.
\end{definition}

\begin{proposition}[Cocharacter lattice of the torus]
  \label{1-1-cochar-torus}
  \uses{1-1-cochar}
  % \lean{}
  % \leanok

  $N = \Hom(M, \Z) \cong \Z^n$. For $u \in N$ we write $\lambda^u$ for the corresponding cocharacter.
\end{proposition}
\begin{proof}
  \uses{0-full-faithful-spec-grp-alg, 1-1-torus-spec}
  % \leanok

  By Propositions \ref{1-1-torus-spec} and \ref{0-full-faithful-spec-grp-alg} in turn, we have
  \[
    \mathrm{cochar}(\Gm^n) = \Hom_{\mathsf{GrpSch}}(\Gm, \Gm^n) = \Hom(k[\Z^n], k[\Z]) = \Hom(\Z^n, \Z) \cong \Z^n.
  \]
\end{proof}

\begin{definition}[The character-cocharacter pairing]
  \label{1-1-char-cochar-pairing}
  \uses{1-1-char, 1-1-cochar}
  \lean{AlgebraicGeometry.Scheme.Over.charPairing}
  \leanok

  Character lattice and one-parameter subgroup pairing.
\end{definition}


\subsection{The Definition of Affine Toric Variety}


\begin{definition}
  \label{1-1-3-aff-tor-var}
  \uses{1-1-torus}
  \lean{AlgebraicGeometry.ToricVariety}
  \leanok

  A {\bf toric variety} is a variety $X$ with
  \begin{itemize}
    \item an open embedding $T := (\bbC^\times)^n \hookrightarrow X$ with dense image
    \item such that the natural action $T \times T \to T$ of the torus on itself extends to an (algebraic) action $T \times X \to X$.
  \end{itemize}
\end{definition}


\subsection{Lattice Points}


\begin{definition}
  \label{1-1-phiA}
  \uses{1-1-char}
  % \lean{}
  % \leanok

  Given a finite set $\MCA = \{a_1, \dotsc, a_s\} \subseteq M$, define $\Phi_\MCA : T \to \mathbb{A}^s$ given by $\Phi_{\mathcal A} (t) = (\chi^{a_1} (t), \dotsc, \chi^{a_s} (t))$.
\end{definition}


\begin{definition}
  \label{1-1-7-ya}
  \uses{1-1-phiA}
  % \lean{}
  % \leanok

  $Y_\MCA$ is the (Zariski) closure of $\im \Phi_\MCA$ in $\mathbb A^s$.
\end{definition}


\begin{proposition}
  \label{1-1-8-aff-tor-var-ya}
  \uses{1-1-7-ya}
  % \lean{}
  % \leanok

  $Y_\MCA$ is a toric variety.
\end{proposition}
\begin{proof}
  \uses{1-1-1-group-hom-subtorus}
  % \leanok

  TODO
\end{proof}

\begin{proposition}
  \label{1-1-8-char-ya}
  \uses{1-1-7-ya}
  % \lean{}
  % \leanok

  The character lattice of the torus of $Y_\MCA$ is $\Z \MCA$.
\end{proposition}
\begin{proof}
  \uses{1-1-char-torus}
  % \leanok

  $\Phi_\MCA: T_N \to \Gm^s$ factors through the torus of $Y_\MCA$.
  The conclusion follows from looking at the corresponding maps of character lattices.
\end{proof}

\subsection{Toric Ideals}


\begin{proposition}
  \label{1-1-9-ideal-ya}
  \uses{1-1-7-ya}
  % \lean{}
  % \leanok

  The ideal of the affine toric variety $Y_\MCA$ is
  \[
    I(Y_\MCA) = \langle x^{\ell_+} - x^{\ell_-} | \ell \in L\rangle
  \]
\end{proposition}
\begin{proof}
  \uses{}
  % \leanok

  See \cite{Cox_2011}.
\end{proof}


\begin{definition}
  \label{1-1-10-lattice-ideal}
  \uses{}
  \lean{AddMonoidAlgebra.monoidIdeal}
  \leanok

  The ideal $I_L = \langle x^\alpha - x^\beta | \alpha, \beta \in \N^s \text{ and } \alpha - \beta \in L\rangle$ is called the {\bf lattice ideal} of the lattice $L \subseteq \Z^s$.

  A toric ideal is a prime lattice ideal.
\end{definition}


\begin{definition}
  \label{1-1-10-toric-ideal}
  \uses{1-1-10-lattice-ideal}
  \lean{AddMonoidAlgebra.IsToricIdeal}
  \leanok
  A {\bf toric ideal} is a prime lattice ideal.
\end{definition}


\begin{proposition}
  \label{1-1-11-toric-ideal-gen-binomial}
  \uses{1-1-10-toric-ideal}
  \lean{AddMonoidAlgebra.isToricIdeal_iff_exists_span_single_sub_single}
  % \leanok

  Proposition 1.1.11: an ideal is toric if and only if it's prime and generated by binomials $x^\alpha - x^\beta$.
\end{proposition}
\begin{proof}
  \uses{1-1-1-subgroup-subtorus, 1-1-9-ideal-ya}
  % \leanok

\end{proof}


\begin{proposition}[The spectrum of an affine monoid algebra is an affine toric variety]
  \label{1-1-14-aff-tor-var-spec-aff-mon-alg}
  \uses{0-aff-mon, 1-1-3-aff-tor-var}
  \lean{AffineToricVarietyFromMonoid.instToricVariety}
  \leanok

  If $S$ is an affine monoid, then $\Spec(\Bbbk[S])$ is an affine toric variety.
\end{proposition}
\begin{proof}
  \uses{0-aff-mon-alg-domain, 0-full-faithful-grp-alg, 1-1-torus-spec}
  % \leanok

  Identify the torus with $\Bbbk[x_1^{\pm1}, \dotsc, x_n^{\pm1}]$ using Lemma \ref{1-1-torus-spec}.
  $i$ induces a morphism $T \to \Spec(\Bbbk[S])$. It's an open embedding as $i$ gives the localization of $\Bbbk[S]$ at $\chi^{a_i}$, so $\im i$ is an affine open. It's dominant as $\Spec(\Bbbk[S])$ is integral and so is irreducible, and $\im i$ is open and nonempty, so dense. The torus action is given by the natural restriction of comultiplication on $\Bbbk[x_1^{\pm1}, \dotsc, x_n^{\pm1}]$ using Proposition \ref{0-full-faithful-grp-alg}.
\end{proof}


\begin{proposition}
  \label{1-1-14-spec-aff-mon-alg-eq-ya}
  \uses{0-aff-mon, 1-1-3-aff-tor-var, 1-1-7-ya}
  % \lean{}
  % \leanok

  If $S$ is an affine monoid and $\mathcal A$ is a finite set generating $S$ as a monoid, then $\Spec(\Bbbk[S]) = Y_{\mathcal A}$.
\end{proposition}
\begin{proof}
  \uses{1-1-8-aff-tor-var-ya, 1-1-14-char-spec-aff-mon-alg}
  % \leanok

  We get a $\Bbbk$-algebra homomorphism $\pi : \Bbbk[x_1, \dotsc, x_s] \to \Bbbk[\Z S]$ given by $\mathcal A$; this induces a morphism $\Phi_{\mathcal A} : T \to \Bbbk^s$. The kernel of $\pi$ is the toric ideal of $Y_{\mathcal A}$ and $\pi$ is clearly surjective, so $Y_{\mathcal A} = \mathbb V(\ker(\pi)) = \Spec(\Bbbk[x_1, \dotsc, x_s] / \ker(\pi)) = \Spec(\bbC[S])$.
\end{proof}


\begin{proposition}[The character lattice of the spectrum of an affine monoid algebra]
  \label{1-1-14-char-spec-aff-mon-alg}
  \uses{1-1-14-aff-tor-var-spec-aff-mon-alg, 1-1-char}
  % \lean{}
  % \leanok

  If $S$ is an affine monoid, then the character lattice of the torus of $\Spec(\Bbbk[S])$ is $\Z S$.
\end{proposition}
\begin{proof}
  \uses{1-1-14-spec-aff-mon-alg-eq-ya}
  % \leanok

  It is what it is.
\end{proof}


\subsection{Equivalence of Constructions}
  

\begin{definition}
  \label{1-1-tor-act-alg}
  \uses{1-1-torus}
  % \lean{}
  % \leanok

  There is a torus action on the semigroup algebra $\bbC[M]$: given $t\in T_N$ and $f\in \bbC[M]$ define
  \[
    t \cdot f = (p \mapsto f(t^{-1}p)).
  \]
\end{definition}


\begin{lemma}
  \label{1-1-16-total-red}
  \uses{1-1-tor-act-alg}
    Let $A \subseteq \bbC[M]$ be a stable subspace, then
    \[
      A = \bigoplus_{\chi^m \in A} \bbC \cdot \chi^m.
    \]
\end{lemma}
\begin{proof}
  \uses{1-1-2-char-eigenspace-direct-sum}
  % \leanok

  TODO
\end{proof}


\begin{theorem}
  \label{thm:1-1-17}
  \uses{0-aff-mon, 1-1-3-aff-tor-var, 1-1-7-ya, 1-1-10-lattice-ideal}
  TFAE:
  \begin{enumerate}
    \item $V$ is an affine toric variety.
    \item $V = Y_{\mathcal A}$ for some finite $\mathcal A$.
    \item $V$ is an affine variety defined by a toric ideal.
    \item $V = \Spec \Bbbk[S]$ for an affine monoid $S$.
  \end{enumerate}
\end{theorem}
\begin{proof}
  \uses{torActOnAlg, 1-1-8-aff-tor-var-ya, 1-1-9-ideal-ya, 1-1-14-aff-tor-var-spec-aff-mon-alg, lmm:1-1-16}
  % \leanok

  TODO
\end{proof}

\section{Cones and Affine Toric Varieties}

% \subsection{Properties of Affine Toric Varieties}


% \chapter{Projective Toric Varieties}

% \section{Lattice Points and Projective Toric Varieties}

% \section{Lattice Points and Polytopes}

% \subsection{Polytopes and Projective Toric Varieties}

% \section{Properties of Projective Toric Varieties}


% \chapter{Normal Toric Varieties}

% \subsection{Fans and Normal Toric Varieties}

% \subsection{The Orbit-Cone Correspondence}

% \input{chapters/03-3-toric-morphisms.tex}
% \input{chapters/03-4-complete-and-proper.tex}

\input{chapters/biblio}
